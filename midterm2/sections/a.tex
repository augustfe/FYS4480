Show that the unperturbed Hamiltonian $\hat{H}_0$ and $\hat{V}$ commute with both the spin projection $\hat{S}_z$ and the total spin $\hat{S}^2$, given by
\begin{equation*}
    \hat{S}_z := \frac{1}{2} \sum_{p\sigma} \sigma a^\dag_{p\sigma} a_{p\sigma} \quad \text{and} \quad \hat{S}^2 := \hat{S}_z^2 + \frac{1}{2}(\hat{S}_ + \hat{S}_ - + \hat{S}_ - \hat{S}_+),
\end{equation*}
where
\begin{equation*}
    \hat{S}_\pm := \sum_{p} a^\dag_{p\pm} a_{p\mp}.
\end{equation*}

This is an important feature of our system that allows us to block-diagonalize the full Hamiltonian.
We will focus on total spin $S=0$.
In this case, it is convenient to define the so-called pair creation and pair annihilation operators
\begin{equation*}
    \hat{P}^{+}_p = a^\dag_{p+} a^\dag_{p-} \quad \text{and} \quad \hat{P}^{-}_p = a_{p-} a_{p+},
\end{equation*}
respectively.

Show that you can rewrite the Hamiltonian (with $\xi=1$) as
\begin{equation*}
    \hat{H} = \sum_{p\sigma} (p-1) a_{p\sigma}^{\dagger} a_{p\sigma} - \frac{1}{2} g \sum_{pq} \hat{P}^{+}_p \hat{P}^{-}_q.
\end{equation*}
Show also that pair creation operators  commute among themselves.

In this midterm we focus only on a system with no broken pairs.
This means that the Hamiltonian can only link two-particle states in so-called spin-reversed states.

\subsection{}
We firstly need to show that the unperturbed Hamiltonian $\hat{H}_0$ and the two-body operator $\hat{V}$ commute with the spin projection $\hat{S}_z$.
We have, being careful with the summation indices,
\begin{align*}
    \left[ \hat{H}_0, \hat{S}_z \right]
    &= \left[
        \sum_{p\sigma} (p-1) a_{p\sigma}^{\dagger} a_{p\sigma},
        \frac{1}{2} \sum_{q\tau} \tau a^\dag_{q\tau} a_{q\tau}
    \right] \\
    &= \frac{1}{2} \sum_{p\sigma} (p-1) \sum_{q\tau} \tau \left[
        a_{p\sigma}^{\dagger} a_{p\sigma},
        a^\dag_{q\tau} a_{q\tau}
    \right] \\
    &= \frac{1}{2} \sum_{p\sigma} (p-1) \sum_{q\tau} \tau \left[
        \hat{n}_{p\sigma}, \hat{n}_{q\tau}
    \right],
\end{align*}
where we have defined the number operator $\hat{n}_{p\sigma} = a_{p\sigma}^{\dagger} a_{p\sigma}$.
As the number operator commutes with itself, we have that $\left[ \hat{n}_{p\sigma}, \hat{n}_{q\tau} \right] = 0$, and thus $[ \hat{H}_0, \hat{S}_z ] = 0$.

Next, we show that the two-body operator $\hat{V}$ commutes with the spin projection $\hat{S}_z$.
We have, again being careful with the summation indices,
\begin{align*}
    \left[ \hat{V}, \hat{S}_z \right]
    &= \left[
        -\frac{1}{2} g \sum_{pq} a_{p+}^\dagger a_{p-}^\dagger a_{q-} a_{q+},
        \frac{1}{2} \sum_{r\sigma} \sigma a^\dag_{r\sigma} a_{r\sigma}
    \right] \\
    &= -\frac{1}{4} g \sum_{pqr \sigma} \sigma \left[
        a_{p+}^\dagger a_{p-}^\dagger a_{q-} a_{q+},
        a^\dag_{r\sigma} a_{r\sigma}
    \right] \\
    &= -\frac{1}{4} g \sum_{pqr \sigma} \sigma \left[
        a_{p+}^\dagger a_{p-}^\dagger a_{q-} a_{q+},
        \hat{n}_{r\sigma}
    \right].
\end{align*}
Using the commutation indentity
\begin{equation*}
    \left[ AB, C \right] = A \left[B, C \right] + \left[ A, C \right] B,
\end{equation*}
we have
\begin{equation}\label{eq:double_comm}
    \left[ a_{p+}^\dagger a_{p-}^\dagger a_{q-} a_{q+}, \hat{n}_{r\sigma} \right]
    = a_{p+}^\dagger a_{p-}^\dagger \Big[ a_{q-} a_{q+}, \hat{n}_{r\sigma} \Big] + \left[ a_{p+}^\dagger a_{p-}^\dagger, \hat{n}_{r\sigma} \right] a_{q-} a_{q+},
\end{equation}
and then need to find expressions for
\begin{equation*}
    \Big[ a_{q-} a_{q+}, \hat{n}_{r\sigma} \Big] \quad \text{and} \quad \left[ a_{p+}^\dagger a_{p-}^\dagger, \hat{n}_{r\sigma} \right].
\end{equation*}

Changing the indices for brevity in the intermediate steps, we need to find
\begin{equation*}
    \left[ a_{p} a_{q}, \hat{n}_{r} \right] \quad \text{and} \quad \left[ a_{p}^\dagger a_{q}^\dagger, \hat{n}_{r} \right],
\end{equation*}
noting that
\begin{equation*}
    \left[ a_{p} a_{q}, \hat{n}_{r} \right]
    = a_{p} \left[ a_{q}, \hat{n}_{r} \right] + \left[ a_{p}, \hat{n}_{r} \right] a_{q}.
\end{equation*}
As
\begin{align*}
    \left[ a_q, \hat{n}_{r} \right] &= \left[ a_q, a_{r}^\dagger a_{r} \right]
    = \left[ a_q, a_{r}^\dagger \right] a_{r} + a_{r}^\dagger \left[ a_q, a_{r} \right] \\
    &= \delta_{qr} a_{r} + a_{r}^\dagger \cdot 0
    = \delta_{qr} a_{r} = a_{q},
\end{align*}
We have
\begin{align*}
    \left[ a_{p} a_{q}, \hat{n}_{r} \right]
    &= a_{p} \left[ a_{q}, \hat{n}_{r} \right] + \left[ a_{p}, \hat{n}_{r} \right] a_{q} \\
    &= a_{p} a_{q} + a_{p} a_{q} = 2 a_{p} a_{q}.
\end{align*}
Similarly, for the creation operators, we have
\begin{equation*}
    \left[ a_p^\dagger, \hat{n}_r \right] = \left[ a_p^\dagger, a_r^\dagger a_r \right] = \left[ a_p^\dagger, a_r^\dagger \right] a_r + a_r^\dagger \left[ a_p^\dagger, a_r \right] = -\delta_{pr} a_r^\dagger = -a_p^\dagger,
\end{equation*}
and thus
\begin{align*}
    \left[ a_p^\dagger a_q^\dagger, \hat{n}_r \right]
    &= a_p^\dagger \left[ a_q^\dagger, \hat{n}_r \right] + \left[ a_p^\dagger, \hat{n}_r \right] a_q^\dagger \\
    &= - a_p^\dagger a_q^\dagger - a_p^\dagger a_q^\dagger = - 2 a_p^\dagger a_q^\dagger.
\end{align*}

Returning to Eq.~\eqref{eq:double_comm} with the correct labels, we have
\begin{align*}
    \left[ a_{p+}^\dagger a_{p-}^\dagger a_{q-} a_{q+}, \hat{n}_{r\sigma} \right]
    &= a_{p+}^\dagger a_{p-}^\dagger \Big[ a_{q-} a_{q+}, \hat{n}_{r\sigma} \Big] + \left[ a_{p+}^\dagger a_{p-}^\dagger, \hat{n}_{r\sigma} \right] a_{q-} a_{q+} \\
    &= 2 a_{p+}^\dagger a_{p-}^\dagger a_{q-} a_{q+} - 2 a_{p+}^\dagger a_{p-}^\dagger a_{q-} a_{q+} \\
    &= 0,
\end{align*}
meaning that
\begin{align*}
    \left[ \hat{V}, \hat{S}_z \right] = -\frac{1}{4} g \sum_{pqr \sigma} \sigma \left[
        a_{p+}^\dagger a_{p-}^\dagger a_{q-} a_{q+},
        \hat{n}_{r\sigma}
    \right] = 0.
\end{align*}
We have thus shown that the unperturbed Hamiltonian $\hat{H}_0$ and the two-body operator $\hat{V}$ commute with the spin projection $\hat{S}_z$.

Next, we want to show the commutations of $\hat{H}_0$ and $\hat{V}$ with the total spin $\hat{S}^2$.
Starting with $\hat{H}_0$, we have
\begin{align*}
    \left[ \hat{H}_0, \hat{S}^2 \right]
    &= \left[
        \hat{H}_0,
        \hat{S}_z^2 + \frac{1}{2}(\hat{S}_+ \hat{S}_- + \hat{S}_- \hat{S}_+)
    \right] \\
    &= \left[ \hat{H}_0, \hat{S}_z^2 \right] + \left[ \hat{H}_0, \frac{1}{2}(\hat{S}_+ \hat{S}_- + \hat{S}_- \hat{S}_+) \right].
\end{align*}
As we have shown that $\hat{H}_0$ commutes with $\hat{S}_z$, we also have $\left[ \hat{H}_0, \hat{S}_z^2 \right] = 0$, and thus only need to place our attention on
\begin{align*}
    \left[ \hat{H}_0, \frac{1}{2}(\hat{S}_+ \hat{S}_- + \hat{S}_- \hat{S}_+) \right]
    &= \frac{1}{2} \left(
        \left[
            \hat{H}_0, \hat{S}_+ \hat{S}_-
        \right]
        + \left[
            \hat{H}_0, \hat{S}_- \hat{S}_+
        \right]
    \right).
\end{align*}
We will only show the commutation of $\hat{H}_0$ with $\hat{S}_+ \hat{S}_-$, as the other commutation is analogous.
Breaking the expression down further, we have
\begin{equation*}
    \left[ \hat{H}_0, \hat{S}_\pm \hat{S}_\mp \right]
    = \left[ \hat{H}_0, \hat{S}_\pm \right] \hat{S}_\mp + \hat{S}_\pm \left[ \hat{H}_0, \hat{S}_\mp \right].
\end{equation*}
We then have
\begin{align*}
    \left[ \hat{H}_0, \hat{S}_{\pm} \right]
    &= \left[
        \sum_{p\sigma} (p-1) a_{p\sigma}^{\dagger} a_{p\sigma},
        \sum_{q} a^\dag_{q\pm} a_{q\mp}
    \right] \\
    &= \sum_{pq \sigma} (p-1) \left[
        a_{p\sigma}^{\dagger} a_{p\sigma},
        a^\dag_{q\pm} a_{q\mp}
    \right].
\end{align*}
Considering the commutation, using the number operator, we have
\begin{align*}
    \left[ \hat{n}_{p\sigma}, a^\dag_{q\pm} a_{q\mp} \right]
    &= a^\dag_{q\pm} \left[ \hat{n}_{p\sigma}, a_{q\mp} \right] + \left[ \hat{n}_{p\sigma}, a^\dag_{q\pm} \right] a_{q\mp} \\
    &= - a^\dag_{q\pm} a_{q\mp} + a^\dag_{q\pm} a_{q\mp} = 0,
\end{align*}
and thus
\begin{equation*}
    \left[ \hat{H}_0, \hat{S}_{\pm} \right] = 0.
\end{equation*}
We have thus shown that $\left[ \hat{H}_0, \hat{S}^2 \right] = 0$.

Next, we show that $\hat{V}$ commutes with $\hat{S}^2$.
We have
\begin{align*}
    \left[ \hat{V}, \hat{S}^2 \right]
    &= \left[
        -\frac{1}{2} g \sum_{pq} a_{p+}^\dagger a_{p-}^\dagger a_{q-} a_{q+},
        \hat{S}_z^2 + \frac{1}{2}(\hat{S}_+ \hat{S}_- + \hat{S}_- \hat{S}_+)
    \right] \\
    &= \left[ \hat{V}, \hat{S}_z^2 \right] + \left[ \hat{V}, \frac{1}{2}(\hat{S}_+ \hat{S}_- + \hat{S}_- \hat{S}_+) \right].
\end{align*}
Again, as we've already shown that $\hat{V}$ commutes with $\hat{S}_z$, we only need to consider the commutation with $\hat{S}_\pm \hat{S}_\mp$.
We analogously again have
\begin{equation*}
    \left[ \hat{V}, \hat{S}_\pm \hat{S}_\mp \right]
    = \left[ \hat{V}, \hat{S}_\pm \right] \hat{S}_\mp + \hat{S}_\pm \left[ \hat{V}, \hat{S}_\mp \right],
\end{equation*}
where
\begin{align*}
    \left[ \hat{V}, \hat{S}_{\pm} \right]
    &= \left[
        -\frac{1}{2} g \sum_{pq} a_{p+}^\dagger a_{p-}^\dagger a_{q-} a_{q+},
        \sum_{r} a^\dag_{r\pm} a_{r\mp}
    \right] \\
    &= -\frac{1}{2} g \sum_{pqr} \left[
        a_{p+}^\dagger a_{p-}^\dagger a_{q-} a_{q+},
        a^\dag_{r\pm} a_{r\mp}
    \right].
\end{align*}
Breaking the expression down further, we have
\begin{equation}\label{eq:V_to_S_comm}
    \begin{split}
        \left[
            a_{p+}^\dagger a_{p-}^\dagger a_{q-} a_{q+},
            a^\dag_{r\pm} a_{r\mp}
        \right] &= a_{p+}^\dagger a_{p-}^\dagger \left[ a_{q-} a_{q+}, a^\dag_{r\pm} a_{r\mp} \right] \\
        &+ \left[ a_{p+}^\dagger a_{p-}^\dagger, a^\dag_{r\pm} a_{r\mp} \right] a_{q-} a_{q+}.
    \end{split}
\end{equation}
Considering the two contractions seperately, starting with the first one:
\begin{equation*}
    \left[ a_{q-} a_{q+}, a^\dag_{r\pm} a_{r\mp} \right]
    = a_{q-} \left[ a_{q+}, a^\dag_{r\pm} a_{r\mp} \right] + \left[ a_{q-}, a^\dag_{r\pm} a_{r\mp} \right] a_{q+}.
\end{equation*}
These expressions follow the same pattern, so we only consider $q \pm \mapsto q$.
\begin{align*}
    \left[ a_{q}, a^\dag_{r\pm} a_{r\mp} \right]
    &= \left[ a_q, a^\dag_{r\pm} \right] a_{r \mp} + a^\dag_{r\pm} \Big[ a_q, a_{r\mp} \Big] \\
    &= \delta_{q, r\pm} a_{r\mp} + a^\dag_{r\pm} \cdot 0 = \delta_{q, r\pm} a_{r\mp}.
\end{align*}
We thus only get a contribution when $q = r$ and the spin parities match.
We then have
\begin{align*}
    \left[ a_{q-} a_{q+}, a^\dag_{r\pm} a_{r\mp} \right]
    &= a_{q-} \left[ a_{q+}, a^\dag_{r\pm} a_{r\mp} \right] + \left[ a_{q-}, a^\dag_{r\pm} a_{r\mp} \right] a_{q+} \\
    &= a_{q-} \delta_{q+, r\pm} a_{r\mp} + \delta_{q-, r\pm} a_{r\mp} a_{q+} \\
    &= \delta_{q+, r\pm} a_{q-} a_{r\mp} - \delta_{q-, r\pm} a_{q+} a_{r\mp} \\
\end{align*}
Continuing with the second contraction, we have
\begin{equation*}
    \left[ a_{p+}^\dagger a_{p-}^\dagger, a^\dag_{r\pm} a_{r\mp} \right] = a_{p+}^\dagger \left[ a_{p-}^\dagger, a^\dag_{r\pm} a_{r\mp} \right] + \left[ a_{p+}^\dagger, a^\dag_{r\pm} a_{r\mp} \right] a_{p-}^\dagger,
\end{equation*}
which again follow the same pattern.
Writing $p \pm \mapsto p$ we have
\begin{align*}
    \left[ a_{p}^\dagger, a^\dag_{r\pm} a_{r\mp} \right]
    &= \left[ a_p^\dagger, a^\dag_{r\pm} \right] a_{r\mp} + a^\dag_{r\pm} \left[ a_p^\dagger, a_{r\mp} \right] \\
    &= 0 \cdot a_{r\mp} - a^\dag_{r\pm} \delta_{p, r\mp} = - a^\dag_{r\pm} \delta_{p, r\mp}.
\end{align*}
Inserting the expressions back into the commutator, we have
\begin{align*}
    \left[ a_{p+}^\dagger a_{p-}^\dagger, a^\dag_{r\pm} a_{r\mp} \right]
    &= a_{p+}^\dagger \left[ a_{p-}^\dagger, a^\dag_{r\pm} a_{r\mp} \right] + \left[ a_{p+}^\dagger, a^\dag_{r\pm} a_{r\mp} \right] a_{p-}^\dagger \\
    &= - a_{p+}^\dagger a^\dag_{r\pm} \delta_{p-, r\mp} - a^\dag_{r\pm} \delta_{p+, r\mp} a_{p-}^\dagger \\
    &= - \delta_{p-, r\mp} a_{p+}^\dagger a^\dag_{r\pm} + \delta_{p+, r\mp} a_{p-}^\dagger a^\dag_{r\pm}.
\end{align*}

As the results are getting quite wieldy, we summarize the different cases of spin parities in the commutator.
\begin{align*}
    \left[ a_{q-} a_{q+}, a^\dag_{r+} a_{r-} \right] &= \delta_{q+, r+} a_{q-} a_{r-} = 0 \\
    \left[ a_{q-} a_{q+}, a^\dag_{r-} a_{r+} \right] &= -\delta_{q-, r-} a_{q+} a_{r+} = 0 \\
    \left[ a_{p+}^\dagger a_{p-}^\dagger, a^\dag_{r+} a_{r-} \right] &= - \delta_{p-, r-} a_{p+}^\dagger a^\dag_{r+} = 0 \\
    \left[ a_{p+}^\dagger a_{p-}^\dagger, a^\dag_{r-} a_{r+} \right] &= \delta_{p+, r+} a_{p-}^\dagger a^\dag_{r-} = 0.
\end{align*}
We thus have that that Eq.~\eqref{eq:V_to_S_comm} simplifies to
\begin{equation*}
    \left[ a_{p+}^\dagger a_{p-}^\dagger a_{q-} a_{q+}, a^\dag_{r\pm} a_{r\mp} \right] = 0,
\end{equation*}
which means that $\left[ \hat{V}, \hat{S}_\pm \right] = 0$, giving us that
\begin{equation*}
    \left[ \hat{V}, \hat{S}^2 \right] = 0.
\end{equation*}
We have thus shown that both $\hat{H}_0$ and $\hat{V}$ commute with the total spin $\hat{S}^2$.

With $\xi = 1$, the one-body operator $\hat{H}_0$ is defined as
\begin{equation*}
    \hat{H}_0 = \sum_{p\sigma} (p-1) a_{p\sigma}^{\dagger} a_{p\sigma}.
\end{equation*}
For the two-body operator $\hat{V}$, we have, substituting $\hat{P}^{+}_p = a_{p+}^\dagger a_{p-}^\dagger$ and $\hat{P}^{-}_q = a_{q-} a_{q+}$,
\begin{equation*}
    \hat{V} = - \frac{1}{2} g \sum_{pq} \hat{P}^{+}_p \hat{P}^{-}_q.
\end{equation*}
This leaves us with the rewritten Hamiltonian
\begin{equation*}
    \hat{H} = \hat{H}_0 + \hat{V} = \sum_{p\sigma} (p-1) a_{p\sigma}^{\dagger} a_{p\sigma} - \frac{1}{2} g \sum_{pq} \hat{P}^{+}_p \hat{P}^{-}_q.
\end{equation*}

Finally, we want to show that the pair creation operators commute among themselves.
\begin{align*}
    \left[ \hat{P}^{+}_p, \hat{P}^{+}_q \right]
    &= \left[ a_{p+}^\dagger a_{p-}^\dagger, a_{q+}^\dagger a_{q-}^\dagger \right] \\
    &= a_{p+}^\dagger a_{p-}^\dagger a_{q+}^\dagger a_{q-}^\dagger - a_{q+}^\dagger a_{q-}^\dagger a_{p+}^\dagger a_{p-}^\dagger \\
    &= a_{p+}^\dagger a_{p-}^\dagger a_{q+}^\dagger a_{q-}^\dagger - (-1)^2 a_{p+}^\dagger a_{q+}^\dagger a_{q-}^\dagger a_{p-}^\dagger \\
    &= a_{p+}^\dagger a_{p-}^\dagger a_{q+}^\dagger a_{q-}^\dagger - (-1)^4 a_{p+}^\dagger a_{p-}^\dagger a_{q+}^\dagger a_{q-}^\dagger \\
    &= 0.
\end{align*}
Similarly, one can show that the pair annihilation operators also commute among themselves.
