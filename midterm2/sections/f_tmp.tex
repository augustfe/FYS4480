
For a second-order contribution to the energy, we get a contribution from the first-order wave operator, which is given by
% The first order wave operator is given by
\begin{equation*}
    \vert \Psi^{(1)} \rangle = \frac{\hat{Q}}{\mathcal{E}_0 - \hat{H}_0} \hat{V} \vert \Phi_0 \rangle = \frac{1}{4} \sum_{\substack{ab \\ ij}} \frac{\langle ab \vert V \vert ij \rangle}{\varepsilon_i + \varepsilon_j - \varepsilon_a - \varepsilon_b} \vert \Phi_{ij}^{ab} \rangle.
\end{equation*}
In exercise 3, we had a block matrix of the from
\begin{equation*}
    \begin{bmatrix}
        \langle \Phi_0 \vert \hat{H} \vert \Phi_0 \rangle & \langle \Phi_0 \vert \hat{H} \vert \Phi_{i\bar{i}}^{a\bar{a}} \rangle \\
        \langle \Phi_{j\bar{j}}^{b\bar{b}} \vert \hat{H} \vert \Phi_0 \rangle & \langle \Phi_{j\bar{j}}^{b\bar{b}} \vert \hat{H} \vert \Phi_{i\bar{i}}^{a\bar{a}} \rangle
    \end{bmatrix}.
\end{equation*}
The possible contributions from the intermediate states generated by the first-order wave operator then correspond with the lower right block of the matrix above, i.e.\ the interactions between two $2p-2h$ states.
% The possible contributions from the first-order wave operator then correspond with the lower right block of the matrix above, i.e. the interactions between two $2p-2h$ states.

The second-order energy contribution is given by
\begin{align*}
    \langle \Phi_0 \vert \hat{V} \vert \Psi^{(1)} \rangle &= \frac{1}{4} \sum_{\substack{ab \\ ij}} \frac{\langle ab \vert V \vert ij \rangle}{\varepsilon_i + \varepsilon_j - \varepsilon_a - \varepsilon_b} \langle \Phi_0 \vert \hat{V} \vert \Phi_{ij}^{ab} \rangle \\
    &= \frac{1}{4} \sum_{\substack{ab \\ ij}} \frac{\langle ij \vert V \vert ab \rangle \langle ab \vert V \vert ij \rangle }{\varepsilon_i + \varepsilon_j - \varepsilon_a - \varepsilon_b} \\
    &= \frac{1}{4} \sum_{ai} \frac{
        \langle i\bar{i} \vert V \vert a\bar{a} \rangle % chktex 7
        \langle a\bar{a} \vert V \vert i\bar{i} \rangle % chktex 7
    }{2(\varepsilon_i - \varepsilon_a)}
\end{align*}
Changing the summation to just sum over the energy levels, we again need to introduce a factor of 2 per spin state, giving
\begin{equation*}
    \frac{1}{2} \sum_{ai} \frac{
        \langle i\bar{i} \vert V \vert a\bar{a} \rangle % chktex 7
        \langle a\bar{a} \vert V \vert i\bar{i} \rangle % chktex 7
    }{i - a} = \frac{1}{8} \sum_{ai} \frac{g^2}{i - a} = -\frac{7}{24} g^2
\end{equation*}

We thus get a total contribution with RSPT to second order of
\begin{equation*}
    2 - g - \frac{7}{24} g^2.
\end{equation*}
