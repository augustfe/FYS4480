\documentclass{article}
\usepackage{graphicx} % Required for inserting images
\usepackage{amsmath}
\usepackage{amsfonts}
\usepackage{comment}
\usepackage{microtype}
\usepackage{bm}

\usepackage{mathtools}
\DeclarePairedDelimiter\bra{\langle}{\rvert}
\DeclarePairedDelimiter\ket{\lvert}{\rangle}
\DeclarePairedDelimiterX\braket[2]{\langle}{\rangle}{#1\,\delimsize\vert\,\mathopen{}#2}
\newcommand*\diff{\mathop{}\!\mathrm{d}}
\DeclarePairedDelimiterX\expval[3]{\langle}{\rangle}%
{#1\,\delimsize\vert\,\mathopen{}#2\,\delimsize\vert\,\mathopen{}#3}

\usepackage{hyperref}
\usepackage{xcolor}
\hypersetup{ % this is just my personal choice, feel free to change things
    colorlinks,
    linkcolor={red!50!black},
    citecolor={blue!50!black},
    urlcolor={blue!80!black},
}

\renewcommand\thesection{Part \alph{section}} % tex-fmt: skip

\title{
    First midterm FYS4480\\
    Quantum mechanics for many-particle systems
}
\author{August Femtehjell}
\date{October 2024}

\begin{document}

\maketitle

\section*{Introduction}

In this midterm we will develop two simple models for studying the helium atom (with two electrons) and the beryllium atom with four electrons.

After having introduced the  Born-Oppenheimer approximation which effectively freezes out the nucleonic degrees of freedom, the Hamiltonian for \(N\) electrons takes the following form
\begin{equation*}
    \hat{H} = \sum_{i=1}^{N} t(x_i) - \sum_{i=1}^{N} k\frac{Ze^2}{r_i} + \sum_{i<j}^{N} \frac{ke^2}{r_{ij}},
\end{equation*}
with \(k=1.44\) eVnm.
Throughout this work we will use atomic units, this means that \(\hbar = c = e = m_e = 1\).
The constant \(k\) becomes also equal 1.
The resulting energies have to be multiplied by \(2 \times 13.6\) eV in order to obtain energies in eletronvolts.

We can rewrite our Hamiltonians as
\begin{equation}
    \hat{H} = \hat{H_0} + \hat{H_I}
    = \sum_{i=1}^{N}\hat{h}_0(x_i) + \sum_{i<j}^{N}\frac{1}{r_{ij}},
    \label{eq:H1H2}
\end{equation}
where  we have defined \(r_{ij} = |\bm{r}_i - \bm{r}_j|\) and \(\hat{h}_0(x_i) =  \hat{t}(x_i) - \frac{Z}{r_i}\).

The variable \(x\) contains both the spatial coordinates and the spin values.
The first term of Eq.~\eqref{eq:H1H2}, \(H_0\), is the sum of the \(N\) \emph{one-body} Hamiltonians \(\hat{h}_0\).
Each individual Hamiltonian \(\hat{h}_0\) contains the kinetic energy operator of an electron and its potential energy due to the attraction of the nucleus.
The second term, \(H_I\), is the sum of the \(N(N-1)/2\) two-body interactions between each pair of electrons.
Note that the double sum carries a restriction \(i<j\).

As basis functions for our calculations we will use hydrogen-like single-particle functions.
This means the onebody operator is diagonal in this basis for states \(i,j\) with quantum numbers
\(n,l,m_l,s,m_s\) with energies
\begin{equation*}
    \expval{i}{\hat{h}_0}{j} = -\frac{Z^2}{2n^2}\delta_{ij}.
\end{equation*}

The quantum number \(n\) refers to the number of nodes of the wave function.
Observe that this expectation value is independent of spin.

We will in all calculations here restrict ourselves to only so-called \(s\)-waves, that is the orbital momentum \(l\) is zero.
We will also limit the quantum number \(n\) to \(n \le 3\).
It means that every \(ns\) state can accommodate two electrons due to the spin degeneracy.

In the calculations you will need the Coulomb interaction with matrix elements involving single-particle wave functions with \(l = 0\) only, the
so-called \(s\)-waves.
We need only the radial part since the spherical harmonics for the \(s\)-waves are rather simple.
We omit single-particle states with \(l > 0\).
The actual integrals we need, are tabulated at the end.
Our radial wave functions are
\begin{equation*}
    R_{n0}(r) = \left( \frac{2Z}{n} \right)^{3/2} \sqrt{\frac{(n-1)!}{2n\times n!}} L_{n-1}^1 \left( \frac{2Zr}{n} \right) \exp{\left( -\frac{Zr}{n} \right)},
\end{equation*}
where \(L_{n-1}^1(r)\) are the so-called Laguerre polynomials.
These wave functions can then be used to compute the direct part of the Coulomb interaction
\begin{equation*}
    \expval*{\alpha\beta}{V}{\gamma\delta} = \int r_1^2 dr_1 \int r_2^2 dr_2 R_{n_{\alpha} 0}^*(r_1) R_{n_{\beta} 0}^*(r_2) \frac{1}{r_{12}}R_{n_{\gamma} 0}(r_1) R_{n_{\delta} 0}(r_2).
\end{equation*}

Observe that this is only the radial integral and that the labels \( \alpha,\beta,\gamma,\delta \) refer only to the quantum numbers \(n,l,m_l\), with \(m_l\) the projection of the orbital momentum \(l\).
A similar expression can be found for the exchange part.
Since we have restricted ourselves to only \(s\)-waves, these integrals are straightforward but tedious to calculate.
As an addendum to this midterm we list all closed-form expressions for the relevant matrix elements.
Note well that these matrix elements do not include spin.
When setting up the final antisymmetrized matrix elements you need to consider the spin degrees of freedom as well.
Please pay in particular attention to the exchange part and the pertinent spin values of the single-particle states.

We will also, for both helium and beryllium assume that the many-particle states we construct have always the same total spin projection \(M_S = 0\).
This means that if we excite one or two particles from the ground state, the spins of the various single-particle states should always sum up to zero.

\section{Setting up the basis}
We start with the helium atom and define our single-particle Hilbert
space to consist of the single-particle orbits \(1s\), \(2s\) and \(3s\),
with their corresponding spin degeneracies.

Set up the ansatz for the ground state \(\ket*{c} = \ket*{\Phi_0}\) in second quantization.
Define the second quantization and define a table of single-particle states.
Construct thereafter all possible one-particle-one-hole
excitations \(\ket*{\Phi_i^a}\) where \(i\) refer to levels below the Fermi level (define this level) and \(a\) refers to particle states.
Define particles and holes.
The Slater determinants have to be written in terms of the respective creation and annihilation operators.
The states you construct should all have total spin projection \(M_S=0\).
Construct also all possible two-particle-two-hole states \(\ket*{\Phi_{ij}^{ab}}\) in a second quantization
representation.

\subsection{}
We define the Fermi level as \(1s\), such that the ground state is given by
\begin{equation}
    \ket*{\Phi_0} = \ket*{c} = a_{1\sigma_1}^\dagger a_{1\sigma_2}^\dagger \ket*{0},
\end{equation}
where we define $\sigma_1 = \ \uparrow \ = +1/2$ and $\sigma_2 = \ \downarrow \ = -1/2$.
Here, we define particles as electrons (?) above the Fermi level, and holes as the lack of electrons in slots below the Fermi level.

In order to have a one-particle-one-hole excitation, the spin in the hole and particle states must match.
All possible one-particle-one-hole (1p1h) excitations are then
\begin{align*}
    \ket*{\Phi_{1\sigma_1}^{2\sigma_1}} &= a_{2\sigma_1}^\dagger a_{1\sigma_1} \ket*{\Phi_0}, &
    \ket*{\Phi_{1\sigma_1}^{3\sigma_1}} &= a_{3\sigma_1}^\dagger a_{1\sigma_1} \ket*{\Phi_0}, \\
    \ket*{\Phi_{1\sigma_2}^{2\sigma_2}} &= a_{2\sigma_2}^\dagger a_{1\sigma_2} \ket*{\Phi_0}, &
    \ket*{\Phi_{1\sigma_2}^{3\sigma_2}} &= a_{3\sigma_2}^\dagger a_{1\sigma_2} \ket*{\Phi_0},
\end{align*}
where we always excite a particle from the $1s$ state, to the higher states, with the same spin such that $M_S = 0$.

For the possible two-particle-two-hole (2p2h) excitations $\ket*{\Phi_{ij}^{ab}}$, we have that both electrons below the Fermi level excite, and that the particles above the Fermi level have opposite spins.
We then have that the possible configurations are
\begin{align*}
    \ket*{\Phi_{1\sigma_1, 1\sigma_2}^{2\sigma_1, 2\sigma_2}} &= a_{2\sigma_1}^\dagger a_{2\sigma_2}^\dagger a_{1\sigma_2} a_{1\sigma_1} \ket*{\Phi_0}, &
    \ket*{\Phi_{1\sigma_1, 1\sigma_2}^{2\sigma_1, 3\sigma_2}} &= a_{2\sigma_1}^\dagger a_{3\sigma_2}^\dagger a_{1\sigma_2} a_{1\sigma_1} \ket*{\Phi_0}, \\
    \ket*{\Phi_{1\sigma_1, 1\sigma_2}^{3\sigma_1, 2\sigma_2}} &= a_{3\sigma_1}^\dagger a_{2\sigma_2}^\dagger a_{1\sigma_2} a_{1\sigma_1} \ket*{\Phi_0}, &
    \ket*{\Phi_{1\sigma_1, 1\sigma_2}^{3\sigma_1, 3\sigma_2}} &= a_{3\sigma_1}^\dagger a_{3\sigma_2}^\dagger a_{1\sigma_2} a_{1\sigma_1} \ket*{\Phi_0}.
\end{align*}



\end{document}
