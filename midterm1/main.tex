\documentclass{article}
\usepackage{graphicx} % Required for inserting images
\usepackage{amsmath}
\usepackage{amsfonts}
\usepackage{comment}
\usepackage{microtype}
\usepackage{bm}
\usepackage{multicol}

\usepackage{mathtools}
\DeclarePairedDelimiter\bra{\langle}{\rvert}
\DeclarePairedDelimiter\ket{\lvert}{\rangle}
\DeclarePairedDelimiterX\braket[2]{\langle}{\rangle}{#1\delimsize\vert\mathopen{}#2}
\newcommand*\diff{\mathop{}\!\mathrm{d}}
\DeclarePairedDelimiterX\expval[3]{\langle}{\rangle}%
{#1\delimsize\vert\mathopen{}#2\delimsize\vert\mathopen{}#3}
\newcommand\HF{\ensuremath{\mathrm{HF}}}

\usepackage{hyperref}
\usepackage{xcolor}
\hypersetup{ % this is just my personal choice, feel free to change things
    colorlinks,
    linkcolor={red!50!black},
    citecolor={blue!50!black},
    urlcolor={blue!80!black},
}

\usepackage{simpler-wick}

\usepackage{enumerate}
\usepackage[shortlabels]{enumitem}

\renewcommand\thesection{Part \alph{section})}  % chktex 9 % chktex 10 % tex-fmt: skip
\renewcommand\thesubsection{Solution} % tex-fmt: skip
\renewcommand\thesubsubsection{\arabic{subsection}.\arabic{subsubsection})} % chktex 9 % chktex 10 % tex-fmt: skip

\title{
    First midterm FYS4480\\
    Quantum mechanics for many-particle systems
}
\author{August Femtehjell \& Oskar Idland}
\date{October 2024}

\begin{document}

\maketitle

\section*{Introduction}

In this midterm we will develop two simple models for studying the helium atom (with two electrons) and the beryllium atom with four electrons.

After having introduced the  Born-Oppenheimer approximation which effectively freezes out the nucleonic degrees of freedom, the Hamiltonian for \(N\) electrons takes the following form
\begin{equation*}
    \hat{H} = \sum_{i=1}^{N} t(x_i) - \sum_{i=1}^{N} k\frac{Ze^2}{r_i} + \sum_{i<j}^{N} \frac{ke^2}{r_{ij}},
\end{equation*}
with \(k=1.44\) eVnm.
Throughout this work we will use atomic units, this means that \(\hbar = c = e = m_e = 1\).
The constant \(k\) becomes also equal 1.
The resulting energies have to be multiplied by \(2 \times 13.6\) eV in order to obtain energies in eletronvolts.

We can rewrite our Hamiltonians as
\begin{equation}
    \hat{H} = \hat{H_0} + \hat{H_I}
    = \sum_{i=1}^{N}\hat{h}_0(x_i) + \sum_{i<j}^{N}\frac{1}{r_{ij}},
    \label{eq:H1H2}
\end{equation}
where  we have defined \(r_{ij} = |\boldsymbol{r}_i - \boldsymbol{r}_j|\) and \(\hat{h}_0(x_i) =  \hat{t}(x_i) - \frac{Z}{r_i}\).

The variable \(x\) contains both the spatial coordinates and the spin values.
The first term of Eq.~\eqref{eq:H1H2}, \(H_0\), is the sum of the \(N\) \emph{one-body} Hamiltonians \(\hat{h}_0\).
Each individual Hamiltonian \(\hat{h}_0\) contains the kinetic energy operator of an electron and its potential energy due to the attraction of the nucleus.
The second term, \(H_I\), is the sum of the \(N(N-1)/2\) two-body interactions between each pair of electrons.
Note that the double sum carries a restriction \(i<j\).

As basis functions for our calculations we will use hydrogen-like single-particle functions.
This means the onebody operator is diagonal in this basis for states \(i,j\) with quantum numbers
\(n,l,m_l,s,m_s\) with energies
\begin{equation}
    \expval{i}{\hat{h}_0}{j} = -\frac{Z^2}{2n^2}\delta_{ij}.
    \label{eq:onebody}
\end{equation}
The quantum number \(n\) refers to the number of nodes of the wave function.
Observe that this expectation value is independent of spin.

We will in all calculations here restrict ourselves to only so-called \(s\)-waves, that is the orbital momentum \(l\) is zero.
We will also limit the quantum number \(n\) to \(n \le 3\).
It means that every \(ns\) state can accommodate two electrons due to the spin degeneracy.

In the calculations you will need the Coulomb interaction with matrix elements involving single-particle wave functions with \(l = 0\) only, the
so-called \(s\)-waves.
We need only the radial part since the spherical harmonics for the \(s\)-waves are rather simple.
We omit single-particle states with \(l > 0\).
The actual integrals we need, are tabulated at the end.
Our radial wave functions are
\begin{equation*}
    R_{n0}(r) = \left( \frac{2Z}{n} \right)^{3/2} \sqrt{\frac{(n-1)!}{2n\times n!}} L_{n-1}^1 \left( \frac{2Zr}{n} \right) \exp{\left( -\frac{Zr}{n} \right)},
\end{equation*}
where \(L_{n-1}^1(r)\) are the so-called Laguerre polynomials.
These wave functions can then be used to compute the direct part of the Coulomb interaction
\begin{equation*}
    \expval*{\alpha\beta}{V}{\gamma\delta} = \int r_1^2 dr_1 \int r_2^2 dr_2 R_{n_{\alpha} 0}^*(r_1) R_{n_{\beta} 0}^*(r_2) \frac{1}{r_{12}}R_{n_{\gamma} 0}(r_1) R_{n_{\delta} 0}(r_2).
\end{equation*}

Observe that this is only the radial integral and that the labels \( \alpha,\beta,\gamma,\delta \) refer only to the quantum numbers \(n,l,m_l\), with \(m_l\) the projection of the orbital momentum \(l\).
A similar expression can be found for the exchange part.
Since we have restricted ourselves to only \(s\)-waves, these integrals are straightforward but tedious to calculate.
As an addendum to this midterm we list all closed-form expressions for the relevant matrix elements.
Note well that these matrix elements do not include spin.
When setting up the final antisymmetrized matrix elements you need to consider the spin degrees of freedom as well.
Please pay in particular attention to the exchange part and the pertinent spin values of the single-particle states.

We will also, for both helium and beryllium assume that the many-particle states we construct have always the same total spin projection \(M_S = 0\).
This means that if we excite one or two particles from the ground state, the spins of the various single-particle states should always sum up to zero.

\section{Setting up the basis}
Show that the unperturbed Hamiltonian $\hat{H}_0$ and $\hat{V}$ commute with both the spin projection $\hat{S}_z$ and the total spin $\hat{S}^2$, given by
\begin{equation*}
    \hat{S}_z := \frac{1}{2} \sum_{p\sigma} \sigma a^\dag_{p\sigma} a_{p\sigma} \quad \text{and} \quad \hat{S}^2 := \hat{S}_z^2 + \frac{1}{2}(\hat{S}_ + \hat{S}_ - + \hat{S}_ - \hat{S}_+),
\end{equation*}
where
\begin{equation*}
    \hat{S}_\pm := \sum_{p} a^\dag_{p\pm} a_{p\mp}.
\end{equation*}

This is an important feature of our system that allows us to block-diagonalize the full Hamiltonian.
We will focus on total spin $S=0$.
In this case, it is convenient to define the so-called pair creation and pair annihilation operators
\begin{equation*}
    \hat{P}^{+}_p = a^\dag_{p+} a^\dag_{p-} \quad \text{and} \quad \hat{P}^{-}_p = a_{p-} a_{p+},
\end{equation*}
respectively.

Show that you can rewrite the Hamiltonian (with $\xi=1$) as
\begin{equation*}
    \hat{H} = \sum_{p\sigma} (p-1) a_{p\sigma}^{\dagger} a_{p\sigma} - \frac{1}{2} g \sum_{pq} \hat{P}^{+}_p \hat{P}^{-}_q.
\end{equation*}
Show also that pair creation operators  commute among themselves.

In this midterm we focus only on a system with no broken pairs.
This means that the Hamiltonian can only link two-particle states in so-called spin-reversed states.

\subsection{}
We firstly need to show that the unperturbed Hamiltonian $\hat{H}_0$ and the two-body operator $\hat{V}$ commute with the spin projection $\hat{S}_z$.
We have, being careful with the summation indices,
\begin{align*}
    \left[ \hat{H}_0, \hat{S}_z \right]
    &= \left[
        \sum_{p\sigma} (p-1) a_{p\sigma}^{\dagger} a_{p\sigma},
        \frac{1}{2} \sum_{q\tau} \tau a^\dag_{q\tau} a_{q\tau}
    \right] \\
    &= \frac{1}{2} \sum_{p\sigma} (p-1) \sum_{q\tau} \tau \left[
        a_{p\sigma}^{\dagger} a_{p\sigma},
        a^\dag_{q\tau} a_{q\tau}
    \right] \\
    &= \frac{1}{2} \sum_{p\sigma} (p-1) \sum_{q\tau} \tau \left[
        \hat{n}_{p\sigma}, \hat{n}_{q\tau}
    \right],
\end{align*}
where we have defined the number operator $\hat{n}_{p\sigma} = a_{p\sigma}^{\dagger} a_{p\sigma}$.
As the number operator commutes with itself, we have that $\left[ \hat{n}_{p\sigma}, \hat{n}_{q\tau} \right] = 0$, and thus $[ \hat{H}_0, \hat{S}_z ] = 0$.

Next, we show that the two-body operator $\hat{V}$ commutes with the spin projection $\hat{S}_z$.
We have, again being careful with the summation indices,
\begin{align*}
    \left[ \hat{V}, \hat{S}_z \right]
    &= \left[
        -\frac{1}{2} g \sum_{pq} a_{p+}^\dagger a_{p-}^\dagger a_{q-} a_{q+},
        \frac{1}{2} \sum_{r\sigma} \sigma a^\dag_{r\sigma} a_{r\sigma}
    \right] \\
    &= -\frac{1}{4} g \sum_{pqr \sigma} \sigma \left[
        a_{p+}^\dagger a_{p-}^\dagger a_{q-} a_{q+},
        a^\dag_{r\sigma} a_{r\sigma}
    \right] \\
    &= -\frac{1}{4} g \sum_{pqr \sigma} \sigma \left[
        a_{p+}^\dagger a_{p-}^\dagger a_{q-} a_{q+},
        \hat{n}_{r\sigma}
    \right].
\end{align*}
Using the commutation indentity
\begin{equation*}
    \left[ AB, C \right] = A \left[B, C \right] + \left[ A, C \right] B,
\end{equation*}
we have
\begin{equation}\label{eq:double_comm}
    \left[ a_{p+}^\dagger a_{p-}^\dagger a_{q-} a_{q+}, \hat{n}_{r\sigma} \right]
    = a_{p+}^\dagger a_{p-}^\dagger \Big[ a_{q-} a_{q+}, \hat{n}_{r\sigma} \Big] + \left[ a_{p+}^\dagger a_{p-}^\dagger, \hat{n}_{r\sigma} \right] a_{q-} a_{q+},
\end{equation}
and then need to find expressions for
\begin{equation*}
    \Big[ a_{q-} a_{q+}, \hat{n}_{r\sigma} \Big] \quad \text{and} \quad \left[ a_{p+}^\dagger a_{p-}^\dagger, \hat{n}_{r\sigma} \right].
\end{equation*}

Changing the indices for brevity in the intermediate steps, we need to find
\begin{equation*}
    \left[ a_{p} a_{q}, \hat{n}_{r} \right] \quad \text{and} \quad \left[ a_{p}^\dagger a_{q}^\dagger, \hat{n}_{r} \right],
\end{equation*}
noting that
\begin{equation*}
    \left[ a_{p} a_{q}, \hat{n}_{r} \right]
    = a_{p} \left[ a_{q}, \hat{n}_{r} \right] + \left[ a_{p}, \hat{n}_{r} \right] a_{q}.
\end{equation*}
As
\begin{align*}
    \left[ a_q, \hat{n}_{r} \right] &= \left[ a_q, a_{r}^\dagger a_{r} \right]
    = \left[ a_q, a_{r}^\dagger \right] a_{r} + a_{r}^\dagger \left[ a_q, a_{r} \right] \\
    &= \delta_{qr} a_{r} + a_{r}^\dagger \cdot 0
    = \delta_{qr} a_{r} = a_{q},
\end{align*}
We have
\begin{align*}
    \left[ a_{p} a_{q}, \hat{n}_{r} \right]
    &= a_{p} \left[ a_{q}, \hat{n}_{r} \right] + \left[ a_{p}, \hat{n}_{r} \right] a_{q} \\
    &= a_{p} a_{q} + a_{p} a_{q} = 2 a_{p} a_{q}.
\end{align*}
Similarly, for the creation operators, we have
\begin{equation*}
    \left[ a_p^\dagger, \hat{n}_r \right] = \left[ a_p^\dagger, a_r^\dagger a_r \right] = \left[ a_p^\dagger, a_r^\dagger \right] a_r + a_r^\dagger \left[ a_p^\dagger, a_r \right] = -\delta_{pr} a_r^\dagger = -a_p^\dagger,
\end{equation*}
and thus
\begin{align*}
    \left[ a_p^\dagger a_q^\dagger, \hat{n}_r \right]
    &= a_p^\dagger \left[ a_q^\dagger, \hat{n}_r \right] + \left[ a_p^\dagger, \hat{n}_r \right] a_q^\dagger \\
    &= - a_p^\dagger a_q^\dagger - a_p^\dagger a_q^\dagger = - 2 a_p^\dagger a_q^\dagger.
\end{align*}

Returning to Eq.~\eqref{eq:double_comm} with the correct labels, we have
\begin{align*}
    \left[ a_{p+}^\dagger a_{p-}^\dagger a_{q-} a_{q+}, \hat{n}_{r\sigma} \right]
    &= a_{p+}^\dagger a_{p-}^\dagger \Big[ a_{q-} a_{q+}, \hat{n}_{r\sigma} \Big] + \left[ a_{p+}^\dagger a_{p-}^\dagger, \hat{n}_{r\sigma} \right] a_{q-} a_{q+} \\
    &= 2 a_{p+}^\dagger a_{p-}^\dagger a_{q-} a_{q+} - 2 a_{p+}^\dagger a_{p-}^\dagger a_{q-} a_{q+} \\
    &= 0,
\end{align*}
meaning that
\begin{align*}
    \left[ \hat{V}, \hat{S}_z \right] = -\frac{1}{4} g \sum_{pqr \sigma} \sigma \left[
        a_{p+}^\dagger a_{p-}^\dagger a_{q-} a_{q+},
        \hat{n}_{r\sigma}
    \right] = 0.
\end{align*}
We have thus shown that the unperturbed Hamiltonian $\hat{H}_0$ and the two-body operator $\hat{V}$ commute with the spin projection $\hat{S}_z$.

Next, we want to show the commutations of $\hat{H}_0$ and $\hat{V}$ with the total spin $\hat{S}^2$.
Starting with $\hat{H}_0$, we have
\begin{align*}
    \left[ \hat{H}_0, \hat{S}^2 \right]
    &= \left[
        \hat{H}_0,
        \hat{S}_z^2 + \frac{1}{2}(\hat{S}_+ \hat{S}_- + \hat{S}_- \hat{S}_+)
    \right] \\
    &= \left[ \hat{H}_0, \hat{S}_z^2 \right] + \left[ \hat{H}_0, \frac{1}{2}(\hat{S}_+ \hat{S}_- + \hat{S}_- \hat{S}_+) \right].
\end{align*}
As we have shown that $\hat{H}_0$ commutes with $\hat{S}_z$, we also have $\left[ \hat{H}_0, \hat{S}_z^2 \right] = 0$, and thus only need to place our attention on
\begin{align*}
    \left[ \hat{H}_0, \frac{1}{2}(\hat{S}_+ \hat{S}_- + \hat{S}_- \hat{S}_+) \right]
    &= \frac{1}{2} \left(
        \left[
            \hat{H}_0, \hat{S}_+ \hat{S}_-
        \right]
        + \left[
            \hat{H}_0, \hat{S}_- \hat{S}_+
        \right]
    \right).
\end{align*}
We will only show the commutation of $\hat{H}_0$ with $\hat{S}_{\pm} \hat{S}_{\mp}$.
Breaking the expression down further, we have
\begin{equation*}
    \left[ \hat{H}_0, \hat{S}_\pm \hat{S}_\mp \right]
    = \left[ \hat{H}_0, \hat{S}_\pm \right] \hat{S}_\mp + \hat{S}_\pm \left[ \hat{H}_0, \hat{S}_\mp \right].
\end{equation*}
We then have
\begin{align*}
    \left[ \hat{H}_0, \hat{S}_{\pm} \right]
    &= \left[
        \sum_{p\sigma} (p-1) a_{p\sigma}^{\dagger} a_{p\sigma},
        \sum_{q} a^\dag_{q\pm} a_{q\mp}
    \right] \\
    &= \sum_{pq \sigma} (p-1) \left[
        a_{p\sigma}^{\dagger} a_{p\sigma},
        a^\dag_{q\pm} a_{q\mp}
    \right].
\end{align*}
Considering the commutation, using the number operator, we have
\begin{align*}
    \left[ \hat{n}_{p\sigma}, a^\dag_{q\pm} a_{q\mp} \right]
    &= a^\dag_{q\pm} \left[ \hat{n}_{p\sigma}, a_{q\mp} \right] + \left[ \hat{n}_{p\sigma}, a^\dag_{q\pm} \right] a_{q\mp} \\
    &= - a^\dag_{q\pm} a_{q\mp} + a^\dag_{q\pm} a_{q\mp} = 0,
\end{align*}
and thus
\begin{equation*}
    \left[ \hat{H}_0, \hat{S}_{\pm} \right] = 0.
\end{equation*}
We have thus shown that $\left[ \hat{H}_0, \hat{S}^2 \right] = 0$.

Next, we show that $\hat{V}$ commutes with $\hat{S}^2$.
We have
\begin{align*}
    \left[ \hat{V}, \hat{S}^2 \right]
    &= \left[
        -\frac{1}{2} g \sum_{pq} a_{p+}^\dagger a_{p-}^\dagger a_{q-} a_{q+},
        \hat{S}_z^2 + \frac{1}{2}(\hat{S}_+ \hat{S}_- + \hat{S}_- \hat{S}_+)
    \right] \\
    &= \left[ \hat{V}, \hat{S}_z^2 \right] + \left[ \hat{V}, \frac{1}{2}(\hat{S}_+ \hat{S}_- + \hat{S}_- \hat{S}_+) \right].
\end{align*}
Again, as we've already shown that $\hat{V}$ commutes with $\hat{S}_z$, we only need to consider the commutation with $\hat{S}_\pm \hat{S}_\mp$.
We analogously again have
\begin{equation*}
    \left[ \hat{V}, \hat{S}_\pm \hat{S}_\mp \right]
    = \left[ \hat{V}, \hat{S}_\pm \right] \hat{S}_\mp + \hat{S}_\pm \left[ \hat{V}, \hat{S}_\mp \right],
\end{equation*}
where
\begin{align*}
    \left[ \hat{V}, \hat{S}_{\pm} \right]
    &= \left[
        -\frac{1}{2} g \sum_{pq} a_{p+}^\dagger a_{p-}^\dagger a_{q-} a_{q+},
        \sum_{r} a^\dag_{r\pm} a_{r\mp}
    \right] \\
    &= -\frac{1}{2} g \sum_{pqr} \left[
        a_{p+}^\dagger a_{p-}^\dagger a_{q-} a_{q+},
        a^\dag_{r\pm} a_{r\mp}
    \right].
\end{align*}
Breaking the expression down further, we have
\begin{equation}\label{eq:V_to_S_comm}
    \begin{split}
        \left[
            a_{p+}^\dagger a_{p-}^\dagger a_{q-} a_{q+},
            a^\dag_{r\pm} a_{r\mp}
        \right] &= a_{p+}^\dagger a_{p-}^\dagger \left[ a_{q-} a_{q+}, a^\dag_{r\pm} a_{r\mp} \right] \\
        &+ \left[ a_{p+}^\dagger a_{p-}^\dagger, a^\dag_{r\pm} a_{r\mp} \right] a_{q-} a_{q+}.
    \end{split}
\end{equation}
Considering the two contractions seperately, starting with the first one:
\begin{equation*}
    \left[ a_{q-} a_{q+}, a^\dag_{r\pm} a_{r\mp} \right]
    = a_{q-} \left[ a_{q+}, a^\dag_{r\pm} a_{r\mp} \right] + \left[ a_{q-}, a^\dag_{r\pm} a_{r\mp} \right] a_{q+}.
\end{equation*}
These expressions follow the same pattern, so we only consider $q \pm \mapsto q$.
\begin{align*}
    \left[ a_{q}, a^\dag_{r\pm} a_{r\mp} \right]
    &= \left[ a_q, a^\dag_{r\pm} \right] a_{r \mp} + a^\dag_{r\pm} \Big[ a_q, a_{r\mp} \Big] \\
    &= \delta_{q, r\pm} a_{r\mp} + a^\dag_{r\pm} \cdot 0 = \delta_{q, r\pm} a_{r\mp}.
\end{align*}
We thus only get a contribution when $q = r$ and the spin parities match.
We then have
\begin{align*}
    \left[ a_{q-} a_{q+}, a^\dag_{r\pm} a_{r\mp} \right]
    &= a_{q-} \left[ a_{q+}, a^\dag_{r\pm} a_{r\mp} \right] + \left[ a_{q-}, a^\dag_{r\pm} a_{r\mp} \right] a_{q+} \\
    &= a_{q-} \delta_{q+, r\pm} a_{r\mp} + \delta_{q-, r\pm} a_{r\mp} a_{q+} \\
    &= \delta_{q+, r\pm} a_{q-} a_{r\mp} - \delta_{q-, r\pm} a_{q+} a_{r\mp} \\
\end{align*}
Continuing with the second contraction, we have
\begin{equation*}
    \left[ a_{p+}^\dagger a_{p-}^\dagger, a^\dag_{r\pm} a_{r\mp} \right] = a_{p+}^\dagger \left[ a_{p-}^\dagger, a^\dag_{r\pm} a_{r\mp} \right] + \left[ a_{p+}^\dagger, a^\dag_{r\pm} a_{r\mp} \right] a_{p-}^\dagger,
\end{equation*}
which again follow the same pattern.
Writing $p \pm \mapsto p$ we have
\begin{align*}
    \left[ a_{p}^\dagger, a^\dag_{r\pm} a_{r\mp} \right]
    &= \left[ a_p^\dagger, a^\dag_{r\pm} \right] a_{r\mp} + a^\dag_{r\pm} \left[ a_p^\dagger, a_{r\mp} \right] \\
    &= 0 \cdot a_{r\mp} - a^\dag_{r\pm} \delta_{p, r\mp} = - a^\dag_{r\pm} \delta_{p, r\mp}.
\end{align*}
Inserting the expressions back into the commutator, we have
\begin{align*}
    \left[ a_{p+}^\dagger a_{p-}^\dagger, a^\dag_{r\pm} a_{r\mp} \right]
    &= a_{p+}^\dagger \left[ a_{p-}^\dagger, a^\dag_{r\pm} a_{r\mp} \right] + \left[ a_{p+}^\dagger, a^\dag_{r\pm} a_{r\mp} \right] a_{p-}^\dagger \\
    &= - a_{p+}^\dagger a^\dag_{r\pm} \delta_{p-, r\mp} - a^\dag_{r\pm} \delta_{p+, r\mp} a_{p-}^\dagger \\
    &= - \delta_{p-, r\mp} a_{p+}^\dagger a^\dag_{r\pm} + \delta_{p+, r\mp} a_{p-}^\dagger a^\dag_{r\pm}.
\end{align*}

As the results are getting quite wieldy, we summarize the different cases of spin parities in the commutator.
\begin{align*}
    \left[ a_{q-} a_{q+}, a^\dag_{r+} a_{r-} \right] &= \delta_{q+, r+} a_{q-} a_{r-} = 0 \\
    \left[ a_{q-} a_{q+}, a^\dag_{r-} a_{r+} \right] &= -\delta_{q-, r-} a_{q+} a_{r+} = 0 \\
    \left[ a_{p+}^\dagger a_{p-}^\dagger, a^\dag_{r+} a_{r-} \right] &= - \delta_{p-, r-} a_{p+}^\dagger a^\dag_{r+} = 0 \\
    \left[ a_{p+}^\dagger a_{p-}^\dagger, a^\dag_{r-} a_{r+} \right] &= \delta_{p+, r+} a_{p-}^\dagger a^\dag_{r-} = 0.
\end{align*}
We thus have that that Eq.~\eqref{eq:V_to_S_comm} simplifies to
\begin{equation*}
    \left[ a_{p+}^\dagger a_{p-}^\dagger a_{q-} a_{q+}, a^\dag_{r\pm} a_{r\mp} \right] = 0,
\end{equation*}
which means that $\left[ \hat{V}, \hat{S}_\pm \right] = 0$, giving us that
\begin{equation*}
    \left[ \hat{V}, \hat{S}^2 \right] = 0.
\end{equation*}
We have thus shown that both $\hat{H}_0$ and $\hat{V}$ commute with the total spin $\hat{S}^2$.

With $\xi = 1$, the one-body operator $\hat{H}_0$ is defined as
\begin{equation*}
    \hat{H}_0 = \sum_{p\sigma} (p-1) a_{p\sigma}^{\dagger} a_{p\sigma}.
\end{equation*}
For the two-body operator $\hat{V}$, we have, substituting $\hat{P}^{+}_p = a_{p+}^\dagger a_{p-}^\dagger$ and $\hat{P}^{-}_q = a_{q-} a_{q+}$,
\begin{equation*}
    \hat{V} = - \frac{1}{2} g \sum_{pq} \hat{P}^{+}_p \hat{P}^{-}_q.
\end{equation*}
This leaves us with the rewritten Hamiltonian
\begin{equation*}
    \hat{H} = \hat{H}_0 + \hat{V} = \sum_{p\sigma} (p-1) a_{p\sigma}^{\dagger} a_{p\sigma} - \frac{1}{2} g \sum_{pq} \hat{P}^{+}_p \hat{P}^{-}_q.
\end{equation*}

Finally, we want to show that the pair creation operators commute among themselves.
\begin{align*}
    \left[ \hat{P}^{+}_p, \hat{P}^{+}_q \right]
    &= \left[ a_{p+}^\dagger a_{p-}^\dagger, a_{q+}^\dagger a_{q-}^\dagger \right] \\
    &= a_{p+}^\dagger a_{p-}^\dagger a_{q+}^\dagger a_{q-}^\dagger - a_{q+}^\dagger a_{q-}^\dagger a_{p+}^\dagger a_{p-}^\dagger \\
    &= a_{p+}^\dagger a_{p-}^\dagger a_{q+}^\dagger a_{q-}^\dagger - (-1)^2 a_{p+}^\dagger a_{q+}^\dagger a_{q-}^\dagger a_{p-}^\dagger \\
    &= a_{p+}^\dagger a_{p-}^\dagger a_{q+}^\dagger a_{q-}^\dagger - (-1)^4 a_{p+}^\dagger a_{p-}^\dagger a_{q+}^\dagger a_{q-}^\dagger \\
    &= 0.
\end{align*}
Similarly, one can show that the pair annihilation operators also commute among themselves.


\section{Second quantized Hamiltonian}
% Define the Hamiltonian in a second-quantized form and use this to compute the expectation value of the ground state (defining the so-called reference energy and later our Hartree-Fock functional) of
the helium atom.
Show that it is given by
\begin{equation}
    E[\Phi_0] = \expval{c}{\hat{H}}{c} = \sum_{i} \expval{i}{\hat{h}_0}{i} + \frac{1}{2} \sum_{ij} \left[\expval*{ij}{\frac{1}{r_{ij}}}{ij} - \expval*{ij}{\frac{1}{r_{ij}}}{ji}\right].
\end{equation}
Define properly the sums keeping in mind that the states $ij$ refer to all quantum numbers $n, l, m_l, s, m_s$.
Use the values for the various matrix elements listed at the end of the midterm to find the value of $E$ as function of $Z$ and compute $E$ as function of $Z$.

\subsection{}
We consider the Hamiltonian $\hat{H} = \hat{H}_0 + \hat{H}_I$, where $\hat{H}_0$ is the one-body part and $\hat{H}_I$ is the two-body part, given by
\begin{align*}
    \hat{H}_0 &= \sum_{i=1}^{N}\hat{h}_0(x_i), &
    \hat{H}_I &= \sum_{i<j}^{N}\frac{1}{r_{ij}}.
\end{align*}
In second quantization, we rewrite the one-body part as
\begin{equation}
    \hat{H}_0 = \sum_{\alpha\beta} \expval{\alpha}{\hat{h}_0}{\beta} a_\alpha^\dagger a_\beta.
\end{equation}
Then, the expectation value of the ground state with the one-body part is given by
\begin{equation*}
    \expval{\Phi_0}{\hat{H}_0}{\Phi_0} = \sum_{\alpha\beta} \expval{\alpha}{\hat{h}_0}{\beta} \expval{\Phi_0}{a_\alpha^\dagger a_\beta}{\Phi_0}.
\end{equation*}
For all states where either $\alpha > F, \beta > F$, we have that $\expval{\Phi_0}{a_\alpha^\dagger a_\beta}{\Phi_0} = 0$.
Thus, the sum is restricted to $i,j \le F$,
\begin{align*}
    \expval{\Phi_0}{\hat{H}_0}{\Phi_0} &= \sum_{ij} \expval{i}{\hat{h}_0}{j} \expval{\Phi_0}{a_i^\dagger a_j}{\Phi_0} \\
    &= \sum_{ij} \expval{i}{\hat{h}_0}{j} \delta_{ij} \\
    &= \sum_{i} \expval{i}{\hat{h}_0}{i},
\end{align*}
where we utilized the orthonormality of the single-particle states.

The two-body part is rewritten in second quantization as
\begin{equation*}
    \hat{H}_I = \frac{1}{2} \sum_{\alpha \beta \gamma \delta} \expval{\alpha \beta}{V}{\gamma \delta} a_\alpha^\dagger a_\beta^\dagger a_\delta a_\gamma.
\end{equation*}
The expectation value of the ground state with the two-body part is then
\begin{equation*}
    \expval{\Phi_0}{\hat{H}_I}{\Phi_0} = \frac{1}{2} \sum_{\alpha\beta\gamma\delta} \expval{\alpha\beta}{V}{\gamma\delta} \expval{\Phi_0}{a_\alpha^\dagger a_\beta^\dagger a_\delta a_\gamma}{\Phi_0}.
\end{equation*}
The possible contributing contractions are
\begin{align*}
    \wick{\c2 a_\alpha^\dagger \c1 a_\beta^\dagger \c1 a_\delta \c2 a_\gamma} &= \delta_{\alpha\gamma} \delta_{\beta\delta}, &
    \wick{\c1 a_\alpha^\dagger \c2 a_\beta^\dagger \c1 a_\delta \c2 a_\gamma} &= -\delta_{\alpha\delta} \delta_{\beta\gamma}. \\
\end{align*}
Whenever $\alpha > F$ or $\beta > F$, the expectation value vanishes, so we relabel the summation to $i, j$. The terms also vanish if $i = j$. % TODO: Reword maybe
We are then left with
\begin{equation*}
    \expval{\Phi_0}{\hat{H}_0}{\Phi_0} = \frac{1}{2} \sum_{\substack{ij \\ i \neq j}} \expval*{ij}{V}{ij} - \expval*{ij}{V}{ji}.
\end{equation*}

Gathering this, we get that the complete expectation value of the ground state is
\begin{equation}
    E[\Phi_0]
    = \expval{c}{\hat{H}}{c}
    = \sum_{i} \expval{i}{\hat{h}_0}{i} + \frac{1}{2} \sum_{\substack{ij \\ i \neq j}} \expval*{ij}{\frac{1}{r_{ij}}}{ij} - \expval*{ij}{\frac{1}{r_{ij}}}{ji},
\end{equation}
as we wanted to show.

In the case of the electrons in the helium atom, we only have $n = 1$, $l = 0$, differing only in the spin quantum number $m_s = \pm 1/2$.
The expectation value of the one-body part is then
\begin{equation*}
    \expval{\Phi_0}{\hat{h}_0}{\Phi_0} = \sum_{\sigma \in \{\pm 1/2\}} \expval{1\sigma}{\hat{h}_0}{1\sigma} = -\frac{Z^2}{n^2},
\end{equation*}
and the expectation value of the two-body part is, writing just $\sigma_{+}$ and $\sigma_{-}$ for the spins,
\begin{equation*}
    \expval{\Phi_0}{\hat{H}_I}{\Phi_0}
    = \frac{1}{2} \sum_{\substack{\sigma_{+}\sigma_{-}\\ \sigma_{+} \neq \sigma_{-}}}
    \underbrace{\expval*{\sigma_{+} \sigma_{-}}{\frac{1}{r_{\sigma_{+}\sigma_{-}}}}{\sigma_{+} \sigma_{-}}}_{\textnormal{Direct term}}
    - \underbrace{\expval*{\sigma_{+} \sigma_{-}}{\frac{1}{r_{\sigma_{+}\sigma_{-}}}}{\sigma_{-}\sigma_{+}}}_{\textnormal{Exchange term}}.
\end{equation*}
The exchange term vanishes since the states are orthogonal, and we are left with the direct term.
We are then just left with
\begin{equation*}
    \expval{\Phi_0}{\hat{H}_I}{\Phi_0} = \frac{1}{2}\left[ \expval*{\sigma_{+} \sigma_{-}}{\frac{1}{r_{\sigma_{+} \sigma_{-}}}}{\sigma_{+} \sigma_{-}} + \expval*{\sigma_{-} \sigma_{+}}{\frac{1}{r_{\sigma_{+} \sigma_{-}}}}{\sigma_{-} \sigma_{+}}\right].
\end{equation*}
As $\hat{H}_I$ is invariant under the change of label $\sigma$, we can simplify this to
\begin{equation*}
    \expval{\Phi_0}{\hat{H}_I}{\Phi_0} = \expval*{\sigma_{+} \sigma_{-}}{\frac{1}{r_{\sigma_{+} \sigma_{-}}}}{\sigma_{+} \sigma_{-}}.
\end{equation*}

Computing this, we find that the expectation value of the ground state is
\begin{equation}
    E[\Phi_0] = -Z^2 + \tfrac{5}{8}Z,
\end{equation}
which as a function of $Z$ is shown in \autoref{fig:energy}.

\begin{figure}[ht]
    \centering
    \includegraphics{figs/energy_plot.pdf}
    \caption{The expectation value of the ground states of an atom with two electrons as a function of the nuclear charge $Z$.\label{fig:energy}}
\end{figure}

Define the Hamiltonian in a second-quantized form and use this to compute the expectation value of the ground state (defining the so-called reference energy and later our Hartree-Fock functional) of
the helium atom.
Show that it is given by
\begin{equation}
    E[\Phi_0] = \expval{c}{\hat{H}}{c} = \sum_{i} \expval{i}{\hat{h}_0}{i} + \frac{1}{2} \sum_{ij} \left[\expval*{ij}{\frac{1}{r_{ij}}}{ij} - \expval*{ij}{\frac{1}{r_{ij}}}{ji}\right].
\end{equation}
Define properly the sums keeping in mind that the states $ij$ refer to all quantum numbers $n, l, m_l, s, m_s$.
Use the values for the various matrix elements listed at the end of the midterm to find the value of $E$ as function of $Z$ and compute $E$ as function of $Z$.

\subsection{}
We consider a Hamiltonian $\hat{H} = \hat{H}_0 + \hat{H}_I$, where $\hat{H}_0$ and $\hat{H}_I$ are one-electron and two-electron parts respectively, defined by
\begin{align}
    \hat{H}_0 &= \sum_{p q} \expval{p}{\hat{h}_0}{q} a_p^\dagger a_q, &
    \hat{H}_I &= \frac{1}{4} \sum_{pqrs} \expval{pq}{\hat{v}}{rs}_{AS} a_p^\dagger a_q^\dagger a_s a_r.
\end{align}

We have the normal-ordered form of annihilation and creation operators, relative to the reference state, where all creation operators are to the left of all annihilation operators.
For example, we have $N[a_p^\dagger a_q] = a_p^\dagger a_q$, $N[a_p a_q^\dagger] = -a_q^\dagger a_p$, where the sign is dependent on the number of permutations required to bring the operators to normal order.
We are interested in this, as
\begin{equation*}
    \expval{c}{N[A B\dotsm]}{c} = 0
\end{equation*}
if $N[A B \dotsm]$ is not empty, where $A, B, \dotsc$ are annihilation or creation operators.

With this, we have the contractions of operators, defined as
\begin{equation*}
    \wick{\c A \c B} = AB - N[AB].
\end{equation*}
Relative to our reference state, we have that
\begin{align*}
    \wick{\c a_i^\dagger \c a_j} &= \delta_{i j}, &
    \wick{\c a_a \c a_b^\dagger} &= \delta_{a b}
\end{align*}
are the only non-zero contractions.

For the one-body term, we then have
\begin{equation}
    \expval{c}{\hat{H}_0}{c} = \sum_{pq} \expval{p}{\hat{h}_0}{q} \expval{c}{a_p^\dagger a_q}{c} = \sum_{ij} \expval{i}{\hat{h}_0}{j} \delta_{ij} = \sum_{i} \expval{i}{\hat{h}_0}{i}.
\end{equation}

For the two-body term, writing $\ket{c} = \ket{i j}$ we first need to examine the possible contractions of $ i j p^\dagger q^\dagger s r j^\dagger i^\dagger$ and the resulting matrix element $\expval{pq}{V}{rs}_{AS}$.
We have
\begin{align*}
    \wick{
        \c2 j
        \c1 i
        \c1 p^\dagger
        \c2 q^\dagger
        \c2 s
        \c1 r
        \c1 i^\dagger
        \c2 j^\dagger
    } &= \delta_{jq} \delta_{ip} \delta_{sj} \delta_{ri} \to \expval{ij}{V}{ij}_{AS}, \\
    \wick{
        \c2 j
        \c1 i
        \c1 p^\dagger
        \c2 q^\dagger
        \c2 s
        \c1 r
        \c2 i^\dagger
        \c1 j^\dagger
    } &= -\delta_{jq} \delta_{ip} \delta_{si} \delta_{rj} \to -\expval{ij}{V}{ji}_{AS}, \\
    \wick{
        \c2 j
        \c1 i
        \c2 p^\dagger
        \c1 q^\dagger
        \c2 s
        \c1 r
        \c2 i^\dagger
        \c1 j^\dagger
    } &= \delta_{jp} \delta_{iq} \delta_{si} \delta_{rj} \to \expval{ji}{V}{ji}_{AS}, \\
    \wick{
        \c2 j
        \c1 i
        \c2 p^\dagger
        \c1 q^\dagger
        \c2 s
        \c1 r
        \c1 i^\dagger
        \c2 j^\dagger
    } &= -\delta_{jp} \delta_{iq} \delta_{sj} \delta_{ri} \to -\expval{ji}{V}{ij}_{AS}.
\end{align*}
As $\expval{\alpha \beta}{V}{\gamma \delta}_{AS} = - \expval{\alpha \beta}{V}{\delta \gamma}_{AS}$ we gather these terms, and inserting for $V$, leaving us with
\begin{equation}
    \expval{c}{\hat{H}_I}{c} = \frac{1}{2} \sum_{ij} \expval{ij}{\frac{1}{r_{ij}}}{ij}_{AS} = \frac{1}{2} \sum_{ij} \expval{ij}{\frac{1}{r_{ij}}}{ij} - \expval{ij}{\frac{1}{r_{ij}}}{ji}.
\end{equation}

Combining this with the one-body term, we have the total reference energy
\begin{equation}
    E[\Phi_0] = \expval{c}{\hat{H}}{c} = \sum_{i} \expval{i}{\hat{h}_0}{i} + \frac{1}{2} \sum_{ij} \expval{ij}{\frac{1}{r_{ij}}}{ij} - \expval{ij}{\frac{1}{r_{ij}}}{ji},
\end{equation}
as we wanted to show.

In the case of the electrons in the helium atom, we only have $n = 1$, $l = 0$, differing only in the spin quantum number $m_s = \pm 1/2$.
The expectation value of the one-body part is then
\begin{equation*}
    \expval{\Phi_0}{\hat{H}_0}{\Phi_0} = \sum_{\sigma \in \{\pm 1/2\}} \expval{1\sigma}{\hat{h}_0}{1\sigma} = -Z^2,
\end{equation*}
and the expectation value of the two-body part is, writing just $\sigma_{+}$ and $\sigma_{-}$ for the spins with $n = 1$,
\begin{equation*}
    \expval{\Phi_0}{\hat{H}_I}{\Phi_0}
    = \frac{1}{2} \sum_{\substack{\sigma_{+}\sigma_{-}\\ \sigma_{+} \neq \sigma_{-}}}
    \underbrace{\expval*{\sigma_{+} \sigma_{-}}{\frac{1}{r_{\sigma_{+}\sigma_{-}}}}{\sigma_{+} \sigma_{-}}}_{\textnormal{Direct term}}
    - \underbrace{\expval*{\sigma_{+} \sigma_{-}}{\frac{1}{r_{\sigma_{+}\sigma_{-}}}}{\sigma_{-}\sigma_{+}}}_{\textnormal{Exchange term}}.
\end{equation*}
The exchange term vanishes since the states are orthogonal, and we are left with the direct term.
We are then just left with
\begin{equation*}
    \expval{\Phi_0}{\hat{H}_I}{\Phi_0} = \frac{1}{2}\left[ \expval*{\sigma_{+} \sigma_{-}}{\frac{1}{r_{\sigma_{+} \sigma_{-}}}}{\sigma_{+} \sigma_{-}} + \expval*{\sigma_{-} \sigma_{+}}{\frac{1}{r_{\sigma_{+} \sigma_{-}}}}{\sigma_{-} \sigma_{+}}\right].
\end{equation*}
As $\hat{H}_I$ is invariant under the change of label $\sigma$, we can simplify this to
\begin{equation*}
    \expval{\Phi_0}{\hat{H}_I}{\Phi_0} = \expval*{\sigma_{+} \sigma_{-}}{\frac{1}{r_{\sigma_{+} \sigma_{-}}}}{\sigma_{+} \sigma_{-}} = \frac{5}{8} Z.
\end{equation*}

Combining this, we find that the expectation value of the ground state is
\begin{equation}
    E[\Phi_0] = -Z^2 + \tfrac{5}{8}Z,
\end{equation}
which as a function of $Z$ is shown in \autoref{fig:energy}.

\begin{figure}[ht]
    \centering
    \includegraphics{figs/energy_plot.pdf}
    \caption{The expectation value of the ground states of an atom with two electrons as a function of the nuclear charge $Z$.\label{fig:energy}}
\end{figure}

\section{Limiting ourselves to one-particle-one excitations}
Hereafter we will limit ourselves to a system which now contains only one-particle-one-hole excitations beyond the chosen state $\ket{c}$.
Using the possible Slater determinants from exercise a) for the helium atom, find the expressions (without inserting the explicit values for the matrix elements first) for % chktex 10 % tex-fmt: skip
\begin{equation*}
    \expval{c}{\hat{H}}{\Phi_i^a},
\end{equation*}
and
\begin{equation*}
    \expval{\Phi_i^a}{\hat{H}}{\Phi_j^b}.
\end{equation*}
Represent these expressions in a diagrammatic form, both for the onebody part and the two-body part of the Hamiltonian.

Insert then the explicit values for the various matrix elements and set up the final Hamiltonian matrix and diagonalize it using for example Python as programming language.
Compare your results from those of exercise b) and comment your results. % chktex 10 % tex-fmt: skip

The exact energy with our Hamiltonian is $-2.9037$ atomic units for helium.
This value is also close to the experimental energy.

\subsection{}
In order to be able to handle the more complicated systems, we partition the Hamiltonian into
\begin{equation}
    \hat{H} = \underbrace{\mathcal{E}_0^{\text{Ref}}}_{\expval{c}{\hat{H}}{c}} + \hat{F}_N + \hat{V}_N,
\end{equation}
where
\begin{align*}
    \hat{F}_N &= \sum_{pq} \expval{p}{f}{q} \{a_p^\dagger a_q\}, \qquad
    \expval{p}{f}{q} = \expval{p}{\hat{h}_0}{q} + \sum_{i} \expval{pi}{V}{qi}_{AS}, \\
    \hat{V}_N &= \frac{1}{4} \sum_{pqrs} \expval{pq}{V}{rs}_{AS} \{a_p^\dagger a_q^\dagger a_s a_r\}.
\end{align*}

Considering then $\expval{c}{\hat{H}}{\Phi_i^a}$, we firstly have $\expval{c}{\mathcal{E}_0^{\text{Ref}}}{\Phi_i^a} = 0$, as $\braket{c}{\Phi_i^a} = 0$.
For the next term, we have
\begin{align*}
    \expval{c}{\hat{F}_N}{\Phi_i^a} &= \sum_{pq} \expval{p}{f}{q} \expval{c}{\{ a_p^\dagger a_q \}}{\Phi_i^a}
    = \sum_{pq} \expval{p}{f}{q} \wick{
        \langle c \vert
        \{ \c2 a_p^\dagger \c1 a_q \}
        \{ \c1 a_a^\dagger \c2 a_i \}
        \vert c \rangle
    } \\
    &= \sum_{pq} \expval{p}{f}{q} \delta_{pi} \delta_{qa}
    = \expval{i}{f}{a} \\
    &= \expval{i}{\hat{h}_0}{a} + \sum_{j} \expval{ij}{V}{aj}_{AS}.
\end{align*}
For the last term, we get
\begin{align*}
    \expval{c}{\hat{V}_N}{\Phi_i^a} &= \frac{1}{4} \sum_{pqrs} \expval{pq}{V}{rs}_{AS} \expval{c}{\{a_p^\dagger a_q^\dagger a_s a_r \}}{\Phi_i^a} \\
    &= \frac{1}{4} \sum_{pqrs} \expval{pq}{V}{rs}_{AS} \wick{
        \langle c \vert
        \{ a_p^\dagger a_q^\dagger a_s a_r \}
        \{ a_a^\dagger a_i \}
        \vert c \rangle
    } \\
    &= 0,
\end{align*}
which vanishes as this would require a contraction within the normal ordered operator $\{a_p^\dagger a_q^\dagger a_s a_r\}$.

Considering next $\expval{\Phi_i^a}{\hat{H}}{\Phi_{j}^{b}}$, we have
\begin{equation*}
    \expval{\Phi_i^a}{\mathcal{E}_0^{\text{Ref}}}{\Phi_{j}^{b}} = \mathcal{E}_0^{\text{Ref}}
    \expval{c}{\wick{
            \{ \c2 a_i^\dagger \c1 a_a \}
            \{ \c1 a_b^\dagger \c2 a_j \}
        }
    }{c}
    = \delta_{ij} \delta_{ab} \mathcal{E}_0^{\text{Ref}}.
\end{equation*}
Next, we have
\begin{align*}
    \expval{\Phi_i^a}{\hat{F}_N}{\Phi_{j}^{b}}
    &= \sum_{pq} \expval{p}{f}{q} \expval{\Phi_i^a}{\{ a_p^\dagger a_q \}}{\Phi_{j}^{b}} \\
    &= \sum_{pq}
    \expval{p}{f}{q}
    \expval{c}{
        \wick{
            \{ a_i^\dagger a_a \}
            \{ a_p^\dagger a_q \}
            \{ a_b^\dagger a_j \}
        }
    }{c}.
\end{align*}
Considering the contractions seperately, we have the two possible contractions
\begin{align*}
    \wick{
        \langle c \vert
        \{ \c2 a_i^\dagger \c1 a_a \}
        \{ \c1 a_p^\dagger \c1 a_q \}
        \{ \c1 a_b^\dagger \c2 a_j \}
        \vert c \rangle
    }
    &= \delta_{ij} \delta_{ap} \delta_{bq}, \\
    \wick{
        \langle c \vert
        \{ \c1 a_i^\dagger \c2 a_a \}
        \{ \c3 a_p^\dagger \c1 a_q \}
        \{ \c2 a_b^\dagger \c3 a_j \}
        \vert c \rangle
    } &= -\delta_{iq} \delta_{ab} \delta_{jp},
\end{align*}
leaving us with
\begin{equation*}
    \expval{\Phi_i^a}{\hat{F}_N}{\Phi_{j}^{b}} = \expval{a}{f}{b} \delta_{ij} - \expval{j}{f}{i} \delta_{ab}.
\end{equation*}
Finally, considering the last term, we have
\begin{align*}
    \expval{\Phi_i^a}{\hat{V}_N}{\Phi_{j}^{b}}
    &= \frac{1}{4} \sum_{pqrs} \expval{pq}{V}{rs}_{AS} \expval{\Phi_i^a}{\{a_p^\dagger a_q^\dagger a_s a_r\}}{\Phi_{j}^{b}} \\
    &= \frac{1}{4} \sum_{pqrs} \expval{pq}{V}{rs}_{AS} \expval{c}{
        \wick{
            \{ a_i^\dagger a_a \}
            \{ a_p^\dagger a_q^\dagger a_s a_r \}
            \{ a_b^\dagger a_j \}
        }
    }{c}.
\end{align*}
Considering the contractions seperately, we have the four possible contractions
\begin{align*}
    \wick{
        \langle c \vert
        \{ \c2 a_i^\dagger \c1 a_a \}
        \{ \c1 a_p^\dagger \c3 a_q^\dagger \c2 a_s \c1 a_r \}
        \{ \c1 a_b^\dagger \c3 a_j \}
        \vert c \rangle
    }
    &= -\delta_{is} \delta_{ap} \delta_{jq} \delta_{br}, \\
    \wick{
        \langle c \vert
        \{ \c2 a_i^\dagger \c1 a_a \}
        \{ \c3 a_p^\dagger \c1 a_q^\dagger \c2 a_s \c1 a_r \}
        \{ \c1 a_b^\dagger \c3 a_j \}
        \vert c \rangle
    } &= \delta_{is} \delta_{aq} \delta_{jp} \delta_{br}, \\
    \wick{
        \langle c \vert
        \{ \c2 a_i^\dagger \c1 a_a \}
        \{ \c1 a_p^\dagger \c3 a_q^\dagger \c1 a_s \c2 a_r \}
        \{ \c1 a_b^\dagger \c3 a_j \}
        \vert c \rangle
    } &= \delta_{ir} \delta_{ap} \delta_{jq} \delta_{bs}, \\
    \wick{
        \langle c \vert
        \{ \c2 a_i^\dagger \c1 a_a \}
        \{ \c3 a_p^\dagger \c1 a_q^\dagger \c1 a_s \c2 a_r \}
        \{ \c1 a_b^\dagger \c3 a_j \}
        \vert c \rangle
    } &= -\delta_{ir} \delta_{aq} \delta_{jp} \delta_{bs}.
\end{align*}
Any contraction between $\{a_i^\dagger a_a\}$ and $\{a_b^\dagger a_j\}$ will vanish, as this would require a contraction within central normal ordered operator.
This leaves us with
\begin{align*}
    \expval{\Phi_i^a}{\hat{V}_N}{\Phi_{j}^{b}}
    &= \frac{1}{4} \sum_{pqrs} \expval{pq}{V}{rs}_{AS} \\
    &\times \Big[
        - \delta_{is} \delta_{ap} \delta_{jq} \delta_{br}
        + \delta_{is} \delta_{aq} \delta_{jp} \delta_{br}
        + \delta_{ir} \delta_{ap} \delta_{jq} \delta_{bs}
        - \delta_{ir} \delta_{aq} \delta_{jp} \delta_{bs}
    \Big],
\end{align*}
which when inserted gives
\begin{align*}
    \expval{\Phi_i^a}{\hat{V}_N}{\Phi_{j}^{b}} &= \frac{1}{4} \big[
        -\expval{aj}{V}{bi}_{AS} + \expval{ja}{V}{bi}_{AS} + \expval{aj}{V}{ib}_{AS} - \expval{ja}{V}{ib}_{AS}
    \big] \\
    &= \frac{1}{4} \big[
        \expval{aj}{V}{ib}_{AS} + \expval{ja}{V}{bi}_{AS} + \expval{aj}{V}{ib}_{AS} + \expval{ja}{V}{bi}_{AS}
    \big] \\
    &= \expval{aj}{V}{ib}_{AS}.
\end{align*}
We have thus shown that
\begin{equation}
    \begin{split}
        \expval{c}{\hat{H}}{\Phi_i^a} &= \expval{i}{\hat{h}_0}{a} + \sum_{j} \expval{ij}{V}{aj}_{AS}, \\
        \expval{\Phi_i^a}{\hat{H}}{\Phi_{j}^{b}} &= \delta_{ij} \delta_{ab} \mathcal{E}_0^{\text{Ref}} + \expval{a}{f}{b} \delta_{ij} - \expval{j}{f}{i} \delta_{ab} + \expval{aj}{V}{ib}_{AS}.
    \end{split}
\end{equation}
% Inserting for $f$, we can simplify the later expressions, as $\hat{h}_0$ defined in Eq.~\eqref{eq:onebody} contains a $\delta_{\alpha \beta}$.
% We then have
% \begin{equation*}
%     \expval{\alpha}{f}{\beta} = \expval{\alpha}{\hat{h}_0}{\beta} + \sum_{k} \expval{\alpha k}{V}{\beta k}_{AS} = \delta_{\alpha \beta} \expval{\alpha}{\hat{h}_0}{\alpha} + \sum_{k} \expval{\alpha k}{V}{\beta k}_{AS}.
% \end{equation*}
% Our final expression for $\expval{\Phi_i^a}{\hat{H}}{\Phi_{j}^{b}}$ then becomes
% \begin{align*}
%     \expval{\Phi_i^a}{\hat{H}}{\Phi_{j}^{b}} &= \delta_{ij} \delta_{ab} \Big[ \mathcal{E}_0^{\text{Ref}} + \expval{a}{\hat{h}_0}{a}  - \expval{i}{\hat{h}_0}{i} \Big] + \expval{aj}{V}{ib}_{AS} \\
%     &+ \sum_{k} \Big[ \delta_{ij} \expval{ak}{V}{bk} - \delta_{ab} \expval{jk}{V}{ik} \Big]
% \end{align*}

% Considering the different cases of $(i, a), (j, b)$ for the helium atom, we have when $i = j$ and $a = b$
% \begin{align*}
%     \expval{\Phi_i^a}{\hat{H}}{\Phi_i^a} &= \mathcal{E}_0^{\text{Ref}} + \expval{a}{\hat{h}_0}{a} - \expval{i}{\hat{h}_0}{i} + \expval{ai}{V}{ia}_{AS}  \\
%     &+ \sum_{k} \Big[ \expval{ak}{V}{ak}_{AS} - \expval{ik}{V}{ik}_{AS} \Big],
% \end{align*}
% When $i = j$ and $a \neq b$
% \begin{equation*}
%     \expval{\Phi_i^a}{\hat{H}}{\Phi_i^b} = \expval{ai}{V}{ib}_{AS} + \sum_k \Big[ \expval{ak}{V}{bk}_{AS} \Big],
% \end{equation*}
% When $i \neq j$ and $a = b$
% \begin{equation*}
%     \expval{\Phi_i^a}{\hat{H}}{\Phi_j^a} = \expval{aj}{V}{ia}_{AS} - \sum_k \Big[ \expval{jk}{V}{ik}_{AS} \Big],
% \end{equation*}
% and finally when $i \neq j$ and $a \neq b$
% \begin{equation*}
%     \expval{\Phi_i^a}{\hat{H}}{\Phi_j^b} = \expval{aj}{V}{ib}_{AS}.
% \end{equation*}

Inserting for the explicit matrix elements, we get that the energy with our Hamiltonian is $-2.8386$ atomic units, or $-77.2112 \ \text{eV}$ for the helium atom.
We see that we have a higher value than the exact energy, which is expected as the true energy serves as a lower bound to the truncated Hamiltonian.
We also see an improvement from our previous results, which stem from the fact that we are truncating at a higher level of excitations.
The energy is computed with the code in \verb|get_energy.py|.


\section{Moving to the Beryllium atom}
We repeat parts b) and c) but now for the beryllium atom. % chktex 10 % tex-fmt: skip

Define the ansatz for $\ket{c}$ and limit yourself again to one-particle-one-hole excitations.
Compute the reference energy $\expval{c}{\hat{H}}{c}$ as function of $Z$.
Thereafter you will need to set up the appropriate Hamiltonian matrix which involves also one-particle-one-hole excitations.
Diagonalize this matrix and compare your eigenvalues with $\expval{c}{\hat{H}}{c}$ as function of $Z$ and comment your results.
The exact energy with our Hamiltonian is $-14.6674$ atomic units for beryllium.
This value is again close to the experimental energy.

With a given energy functional, we can perform at least two types of variational strategies.
These are:
\begin{enumerate}
    \item Vary the Slater determinant by changing the spatial part of the single-particle wave functions themselves, or

    \item Expand the single-particle functions in a known basis  and vary the coefficients, that is, the new function single-particle wave function $\ket{p}$ is written as a linear expansion in terms of a fixed basis $\phi$ (harmonic oscillator, Laguerre polynomials etc)
        \begin{equation*}
            \psi_p  = \sum_{\lambda} C_{p\lambda}\phi_{\lambda}
        \end{equation*}
\end{enumerate}
Both cases lead to a new Slater determinant which is related to the previous via a unitary transformation.
Below we will set up the Hartree-Fock equations using the second option.
We assume that our basis is still formed by the hydrogen-like wave functions.
We consider a Slater determinant built up of single-particle orbitals $\phi_{\lambda}$ where the indices $\lambda$ refer to specific single-particle states.
As an example, you could think of the ground state ansatz for the beryllium atom.

The unitary transformation
\begin{equation*}
    \psi_p  = \sum_{\lambda} C_{p\lambda}\phi_{\lambda},
\end{equation*}
brings us into the new basis $\psi$.
The new basis is orthonormal and $C$ is a unitary matrix.


\section{Hartree-Fock}
\subsection*{Preamble}
With a given energy functional, we can perform at least two types of variational strategies.
These are:
\begin{enumerate}
    \item Vary the Slater determinant by changing the spatial part of the single-particle wave functions themselves, or

    \item Expand the single-particle functions in a known basis  and vary the coefficients, that is, the new function single-particle wave function $\ket{p}$ is written as a linear expansion in terms of a fixed basis $\phi$ (harmonic oscillator, Laguerre polynomials etc)
        \begin{equation*}
            \psi_p  = \sum_{\lambda} C_{p\lambda}\phi_{\lambda}
        \end{equation*}
\end{enumerate}
Both cases lead to a new Slater determinant which is related to the previous via a unitary transformation.
Below we will set up the Hartree-Fock equations using the second option.
We assume that our basis is still formed by the hydrogen-like wave functions.
We consider a Slater determinant built up of single-particle orbitals $\phi_{\lambda}$ where the indices $\lambda$ refer to specific single-particle states.
As an example, you could think of the ground state ansatz for the beryllium atom.

The unitary transformation
\begin{equation*}
    \psi_p  = \sum_{\lambda} C_{p\lambda}\phi_{\lambda},
\end{equation*}
brings us into the new basis $\psi$.
The new basis is orthonormal and $C$ is a unitary matrix.

\subsection*{Problem definition}
Minimizing with respect to $C^*_{p\alpha}$, remembering that $C^*_{p\alpha}$ and $C_{p\alpha}$ (and that the indices contain all single-particle quantum numbers including spin) are independent and defining
\begin{equation*}
    h_{\alpha\gamma}^{HF} = \expval{\alpha}{h}{\gamma} + \sum_{p} \sum_{\beta\delta} C^*_{p\beta} C_{p\delta} \expval{\alpha\beta}{V}{\gamma\delta}_{AS},
\end{equation*}
show that you can write the Hartree-Fock equations as
\begin{equation*}
    \sum_{\gamma} h_{\alpha\gamma}^{HF} C_{p\gamma} = \epsilon_p^{\mathrm{HF}} C_{p\alpha}. \label{eq:newhf}
\end{equation*}
Explain the meaning of the different terms and define the Hartree-Fock operator in second quantization.
Write down its diagrammatic representation as well.
The greek letters refer to the wave functions in the original basis (in our case the hydrogen-like wave functions) while roman letters refer to the new basis.

\subsection{}



\end{document} % chktex 17
