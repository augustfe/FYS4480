We repeat parts b) and c) but now for the beryllium atom. % chktex 10 % tex-fmt: skip

Define the ansatz for $\ket{c}$ and limit yourself again to one-particle-one-hole excitations.
Compute the reference energy $\expval{c}{\hat{H}}{c}$ as function of $Z$.
Thereafter you will need to set up the appropriate Hamiltonian matrix which involves also one-particle-one-hole excitations.
Diagonalize this matrix and compare your eigenvalues with $\expval{c}{\hat{H}}{c}$ as function of $Z$ and comment your results.
The exact energy with our Hamiltonian is $-14.6674$ atomic units for beryllium.
This value is again close to the experimental energy.

With a given energy functional, we can perform at least two types of variational strategies.
These are:
\begin{enumerate}
    \item Vary the Slater determinant by changing the spatial part of the single-particle wave functions themselves, or

    \item Expand the single-particle functions in a known basis  and vary the coefficients, that is, the new function single-particle wave function $\ket{p}$ is written as a linear expansion in terms of a fixed basis $\phi$ (harmonic oscillator, Laguerre polynomials etc)
        \begin{equation*}
            \psi_p  = \sum_{\lambda} C_{p\lambda}\phi_{\lambda}
        \end{equation*}
\end{enumerate}
Both cases lead to a new Slater determinant which is related to the previous via a unitary transformation.
Below we will set up the Hartree-Fock equations using the second option.
We assume that our basis is still formed by the hydrogen-like wave functions.
We consider a Slater determinant built up of single-particle orbitals $\phi_{\lambda}$ where the indices $\lambda$ refer to specific single-particle states.
As an example, you could think of the ground state ansatz for the beryllium atom.

The unitary transformation
\begin{equation*}
    \psi_p  = \sum_{\lambda} C_{p\lambda}\phi_{\lambda},
\end{equation*}
brings us into the new basis $\psi$.
The new basis is orthonormal and $C$ is a unitary matrix.
