The final stage is to set up an iterative scheme where you use the new wave functions determined via the coefficients $C_{p\alpha}$ to solve iteratively the Hartree-Fock equations till a given self-consistency is reached.
A typical way of doing this is to compare the single-particle energies from the previous iteration with those obtained from the new diagonalization.
If the total difference is smaller than a prefixed value, the iterative process is stopped.
Feel free to use the code presented in the lecture notes of week 40.
Compare these results with the those you obtained under the minimization of the ground states as functions of $Z$ and the full diagonalization.
Discuss your results.

\subsection{}
Setting up the iterative scheme now simply amounts to repeating the diagonalization process implemented previously, until convergence is reached.
We measure converge by checking the condition
\begin{equation*}
    \frac{\lvert \varepsilon^{(n)} - \varepsilon^{(n-1)} \rvert}{m} \leq \lambda,
\end{equation*}
where $\varepsilon^{(n)}$ is the new ground state energy, $\varepsilon^{(n-1)}$ is the previous ground state energy, $m$ is the number of single-particle states, $\lambda$ is a predefined tolerance, and $\lvert \cdot \rvert$ denotes the 1-norm.
We set $\lambda = 10^{-14}$.

The code for this can again be found in \verb|hartree_fock.py|, and the results are listed in \autoref{tab:energies_converged}.
The untruncated values can be found in \verb|hartree_fock.txt|.
Here, we see only marginal improvements as opposed to the results from the first iteration.

\begin{table}[h!]
    \caption{Single-particle energies and new ground state energies for Helium and Beryllium atoms, after one Hartree-Fock iteration.\label{tab:energies_converged}}
    \centering
    \small
    \begin{tabular}{lcr}
        \toprule\toprule
        Atom & Iterations & Final Energy \\ \midrule
        Helium & 17 & $-2.8310$ \\ \midrule
        Beryllium & 18 & $-14.5082$ \\ \bottomrule
    \end{tabular}
\end{table}

For the helium atom we found the best results when diagonalizing the Hamiltonian directly.
This is unsurprising, given the simpler nature of the problem.
This approach does not however scale well to larger systems, where the iterative scheme is more efficient.
We see tendencies of this already in the beryllium atom, where the iterative scheme gives the best results.
When setting up the Hamiltonian for the beryllium atom, we truncated the excitations at one level above the Fermi level, as opposed to two levels as with the helium atom.
