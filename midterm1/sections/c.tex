Hereafter we will limit ourselves to a system which now contains only one-particle-one-hole excitations beyond the chosen state $\ket{c}$.
Using the possible Slater determinants from exercise a) for the helium atom, find the expressions (without inserting the explicit values for the matrix elements first) for % chktex 10 % tex-fmt: skip
\begin{equation*}
    \expval{c}{\hat{H}}{\Phi_i^a},
\end{equation*}
and
\begin{equation*}
    \expval{\Phi_i^a}{\hat{H}}{\Phi_j^b}.
\end{equation*}
Represent these expressions in a diagrammatic form, both for the onebody part and the two-body part of the Hamiltonian.

Insert then the explicit values for the various matrix elements and set up the final Hamiltonian matrix and diagonalize it using for example Python as programming language.
Compare your results from those of exercise b) and comment your results. % chktex 10 % tex-fmt: skip

The exact energy with our Hamiltonian is $-2.9037$ atomic units for helium.
This value is also close to the experimental energy.

\subsection{}
We start by finding the expectation value of the Hamiltonian between the ground state $\ket{c}$ and a one-particle-one-hole excitation $\ket{\Phi_i^a}$.
Writing out the terms, we have
\begin{align*}
    \expval{c}{\hat{H}}{\Phi_i^a} &= \expval{c}{\hat{H}_0 + \hat{H}_I}{\Phi_i^a} \\
    &= \expval{c}{\hat{H}_0}{\Phi_i^a} + \expval{c}{\hat{H}_I}{\Phi_i^a}. \\
\end{align*}

Considering the one-body part, we have
\begin{align*}
    \expval{c}{\hat{H}_0}{\Phi_i^a} &= \sum_{\alpha\beta} \expval{\alpha}{\hat{h}_0}{\beta} \expval{c}{a_\alpha^\dagger a_\beta}{\Phi_i^a} \\
    &= \sum_{\alpha\beta} \expval{\alpha}{\hat{h}_0}{\beta} \expval{c}{a_\alpha^\dagger a_\beta a_a^\dagger a_i}{c}. \\
\end{align*}
The only possible contraction is
\begin{equation*}
    \langle
    \wick{
        c
        \vert
        \c2 a_\alpha^\dagger \c1 a_\beta \c1 a_a^\dagger \c2 a_i
        \vert
        c
    }
    \rangle,
\end{equation*}
such that $\alpha = i$ and $\beta = a$.
The expectation value of the one-body part is then
\begin{equation*}
    \expval{c}{\hat{H}_0}{\Phi_i^a} = \sum_{\alpha \beta} \expval{\alpha}{\hat{h}_0}{\beta} \delta_{\alpha i} \delta_{\beta a} = \expval{i}{\hat{h}_0}{a} = 0,
\end{equation*}
which vanishes due to the $\delta_{ij}$ term of Eq.~\eqref{eq:onebody}.

For the two-body part, writing $V$ for the two-particle operator, we have
\begin{align*}
    \expval{c}{\hat{H}_I}{\Phi_i^a} &= \frac{1}{2} \sum_{\alpha\beta\gamma\delta} \expval{\alpha\beta}{V}{\gamma\delta} \expval{c}{a_\alpha^\dagger a_\beta^\dagger a_\delta a_\gamma}{\Phi_i^a} \\
    &= \frac{1}{2} \sum_{\alpha\beta\gamma\delta} \expval{\alpha\beta}{V}{\gamma\delta} \expval{c}{a_\alpha^\dagger a_\beta^\dagger a_\delta a_\gamma a_a^\dagger a_i}{c}.
\end{align*}
The possible contractions are then
\begin{align*}
    \expval{\alpha\beta}{V}{\gamma\delta} \langle
    \wick{
        c
        \vert
        \c3 a_\alpha^\dagger \c2 a_\beta^\dagger \c1 a_\delta \c3 a_\gamma \c1 a_a^\dagger \c2 a_i
        \vert
        c
    }
    \rangle
    &= \delta_{\alpha \gamma} \delta_{\beta i} \delta_{\delta a} \expval{\alpha\beta}{V}{\gamma\delta}
    = \expval{\alpha i}{V}{\alpha a}, \\
    \expval{\alpha\beta}{V}{\gamma\delta} \langle
    \wick{
        c
        \vert
        \c3 a_\alpha^\dagger \c2 a_\beta^\dagger \c3 a_\delta \c1 a_\gamma \c1 a_a^\dagger \c2 a_i
        \vert
        c
    }
    \rangle
    &= -\delta_{\alpha \delta} \delta_{\beta i} \delta_{\gamma a} \expval{\alpha\beta}{V}{\gamma\delta} = -\expval{\alpha i}{V}{a \alpha}, \\
    \expval{\alpha\beta}{V}{\gamma\delta} \langle
    \wick{
        c
        \vert
        \c3 a_\alpha^\dagger \c2 a_\beta^\dagger \c1 a_\delta \c2 a_\gamma \c1 a_a^\dagger \c3 a_i
        \vert
        c
    }
    \rangle
    &= -\delta_{\alpha i} \delta_{\beta \gamma} \delta_{a \delta} \expval{\alpha\beta}{V}{\gamma\delta} = -\expval{i \beta}{V}{\beta a}, \\
    \expval{\alpha\beta}{V}{\gamma\delta} \langle
    \wick{
        c
        \vert
        \c2 a_\alpha^\dagger \c1 a_\beta^\dagger \c1 a_\delta \c1 a_\gamma \c1 a_a^\dagger \c2 a_i
        \vert
        c
    }
    \rangle
    &= \delta_{\alpha i} \delta_{\beta \delta} \delta_{\gamma a} \expval{\alpha\beta}{V}{\gamma\delta} = \expval{i \beta}{V}{a \beta}.
\end{align*}
Using the general fact that $\expval{\alpha \beta}{V}{\gamma \delta} = \expval{\beta \alpha}{V}{\delta \gamma}$, we can gather these terms into a single term
% TODO: FIX OSKAR, gjør greia di
% TODO: Commutation relations
% TODO: Diagrams
\begin{equation*}
    \expval{c}{\hat{H}_I}{\Phi_i^a} = \sum_{\alpha} \expval{\alpha i}{V}{\alpha a} - \expval{\alpha i}{V}{a \alpha}.
\end{equation*}

The final expression for the expectation value of the Hamiltonian between the ground state and a one-particle-one-hole excitation is then
\begin{equation*}
    \expval{c}{\hat{H}}{\Phi_i^a} = \expval{c}{\hat{H}_0}{\Phi_i^a} + \expval{c}{\hat{H}_I}{\Phi_i^a} = 0 + \sum_{\alpha} \expval{\alpha i}{V}{\alpha a} - \expval{\alpha i}{V}{a \alpha}.
\end{equation*}

Inserting the values for all 1p1h excitations:

\subsubsection*{a) $\expval{c}{\hat{H}}{\Phi_{1\sigma_{+}}^{2\sigma_{+}}}$} % chktex 9 % chktex 10 % tex-fmt: skip
The only value that $\alpha$ can be, is $1\sigma_{-}$.
\begin{equation*}
    \expval{1\sigma_{-} 1\sigma_{+}}{V}{1\sigma_{-} 2\sigma_{+}} - \expval{1\sigma_{-} 1\sigma_{+}}{V}{2\sigma_{+} 1\sigma_{-}}
\end{equation*}

\subsubsection*{b) $\expval{c}{\hat{H}}{\Phi_{1\sigma_{+}}^{3\sigma_{+}}}$} % chktex 9 % chktex 10 % tex-fmt: skip
The only value that $\alpha$ can be, is $1\sigma_{-}$.
\begin{equation*}
    \expval{1\sigma_{-} 1\sigma_{+}}{V}{1\sigma_{-} 3\sigma_{+}} - \expval{1\sigma_{-} 1\sigma_{+}}{V}{3\sigma_{+} 1\sigma_{-}}
\end{equation*}

\subsubsection*{c) $\expval{c}{\hat{H}}{\Phi_{1\sigma_{-}}^{2\sigma_{-}}}$} % chktex 9 % chktex 10 % tex-fmt: skip
The only value that $\alpha$ can be, is $1\sigma_{+}$.
\begin{equation*}
    \expval{1\sigma_{+} 1\sigma_{-}}{V}{1\sigma_{+} 2\sigma_{-}} - \expval{1\sigma_{+} 1\sigma_{-}}{V}{2\sigma_{-} 1\sigma_{+}}
\end{equation*}

\subsubsection*{d) $\expval{c}{\hat{H}}{\Phi_{1\sigma_{-}}^{3\sigma_{-}}}$} % chktex 9 % chktex 10 % tex-fmt: skip
The only value that $\alpha$ can be, is $1\sigma_{+}$.
\begin{equation*}
    \expval{1\sigma_{+} 1\sigma_{-}}{V}{1\sigma_{+} 3\sigma_{-}} - \expval{1\sigma_{+} 1\sigma_{-}}{V}{3\sigma_{-} 1\sigma_{+}}
\end{equation*}

