Hereafter we will limit ourselves to a system which now contains only one-particle-one-hole excitations beyond the chosen state $\ket{c}$.
Using the possible Slater determinants from exercise a) for the helium atom, find the expressions (without inserting the explicit values for the matrix elements first) for % chktex 10 % tex-fmt: skip
\begin{equation*}
    \expval{c}{\hat{H}}{\Phi_i^a},
\end{equation*}
and
\begin{equation*}
    \expval{\Phi_i^a}{\hat{H}}{\Phi_j^b}.
\end{equation*}
Represent these expressions in a diagrammatic form, both for the onebody part and the two-body part of the Hamiltonian.

Insert then the explicit values for the various matrix elements and set up the final Hamiltonian matrix and diagonalize it using for example Python as programming language.
Compare your results from those of exercise b) and comment your results. % chktex 10 % tex-fmt: skip

The exact energy with our Hamiltonian is $-2.9037$ atomic units for helium.
This value is also close to the experimental energy.

\subsection{}
We start by finding the value of the expression $\expval{c}{\hat{H}}{\Phi_i^a}$.
Writing out the terms, we have
\begin{align*}
    \expval{c}{\hat{H}}{\Phi_i^a} &= \expval{c}{\hat{H}_0 + \hat{H}_I}{\Phi_i^a} \\
    &= \expval{c}{\hat{H}_0}{\Phi_i^a} + \expval{c}{\hat{H}_I}{\Phi_i^a}.
\end{align*}
We can now read from Eq.~\eqref{eq:H0_second_quant} that the one-body part of the expression vanishes, either through an unmatched $b_\alpha^\dagger$ or simply $\braket{c}{\Phi_i^a} = 0$.
This gives us
\begin{equation}\label{eq:1p1h-ground-onebody}
    \expval{c}{\hat{H}_0}{\Phi_i^a} = 0.
\end{equation}

In order to make sure the number of annihilation and creation terms are correct, the contributing terms from $H_I$ are those with two more annihilation operators than creation operators, while also matching the number of holes and particles created.
We can then quickly reduce the possible contributing candidates to
\begin{equation*}
    \frac{1}{4} \sum_{aijk} \expval{ji}{V}{ak} b_k^\dagger b_j b_i b_a, \qquad
    \frac{1}{2} \sum_{aij} \expval{ji}{V}{ai} b_j b_a, \qquad
    \text{and} \qquad \mathcal{E}_I^\text{Ref}.
\end{equation*}
On closer inspection, we see that the first term vanishes, as $b_k^\dagger$ is unmatched, while the last term vanished due to $\braket{c}{\Phi_i^a} = 0$.

We then just need to evaluate the contractions of the second term, changing the labels of the sum $(a, i, j) \mapsto (b, j, k)$ to avoid confusion with $\ket{\Phi_i^a}$, which simply is
\begin{equation*}
    \langle c \vert
    \wick{
        \c2 b_k \c1 b_b \c1 b_a^\dagger \c2 b_i^\dagger
    }
    \vert c \rangle
    = \delta_{ik} \delta_{ab}.
\end{equation*}
We then get
\begin{equation}\label{eq:1p1h-ground-twobody}
    \expval*{c}{\hat{H}_I}{\Phi_i^a} = \frac{1}{2} \sum_{j} \expval{i j}{V}{a j}_{AS},
\end{equation}
taking into account that the matrix element in the sum is antisymmetrized.
Combining Eqs.~\eqref{eq:1p1h-ground-onebody}~and~\eqref{eq:1p1h-ground-twobody}, we get the final expression is
\begin{equation}\label{eq:1p1h-ground}
    \expval{c}{\hat{H}}{\Phi_i^a} = \frac{1}{2} \sum_{j \neq i} \expval{i j}{V}{a j} - \expval{i j}{V}{j a}.
\end{equation}

Finally, taking into account spin, the values we get for the one-particle-one-hole excitations are listed in Eq.~\eqref{eq:1p1h-ground-list}, with the only possible value for $j$ in each case of Eq.~\eqref{eq:1p1h-ground} listed to the far left.
For brevity, we write $j\sigma_{\pm}$ as just $j_{\pm}$ here.
\begin{equation}\label{eq:1p1h-ground-list}
    \arraycolsep=1.4pt
    \begin{array}{crcl}
        j = 1_{-}: \quad
        & \expval*{c}{\hat{H}}{\Phi_{1_{+}}^{2_{+}}}\hspace{0.27pt}
        & = &
        \expval{1_{-} 1_{+}}{V}{1_{-} 2_{+}} - \expval{1_{-} 1_{+}}{V}{2_{+} 1_{-}} \\

        j = 1_{-}: \quad
        & \expval*{c}{\hat{H}}{\Phi_{1_{+}}^{2_{+}}}\hspace{0.27pt}
        & = &
        \expval{1_{-} 1_{+}}{V}{1_{-} 3_{+}} - \expval{1_{-} 1_{+}}{V}{3_{+} 1_{-}} \\

        j = 1_{+}: \quad
        & \expval*{c}{\hat{H}}{\Phi_{1_{-}}^{2_{-}}}
        & = &
        \expval{1_{+} 1_{-}}{V}{1_{+} 2_{-}} - \expval{1_{+} 1_{-}}{V}{2_{-} 1_{+}} \\

        j = 1_{+}: \quad
        & \expval*{c}{\hat{H}}{\Phi_{1_{-}}^{3_{-}}}
        & = &
        \expval{1_{+} 1_{-}}{V}{1_{+} 3_{-}} - \expval{1_{+} 1_{-}}{V}{3_{-} 1_{+}}
    \end{array}
\end{equation}

Next, we find a simplified expression for $\expval{\Phi_i^a}{\hat{H}}{\Phi_j^b}$,
noting that
\begin{equation*}
    \expval{\Phi_i^a}{\hat{H}}{\Phi_j^b} = \expval{c}{b_i b_a \hat{H} b_b^\dagger b_j^\dagger}{c}.
\end{equation*}
Beginning again with the one-body part of the Hamiltonian, we have can read from Eq.~\eqref{eq:H0_second_quant_simple} that if $(i, a) \neq (j, b)$, the expression vanishes.
If however the pair is equal, we get a contribution, giving us
\begin{equation*}
    \expval{\Phi_i^a}{\hat{H}_0}{\Phi_j^b} = \delta_{ij} \delta_{ab} \left[ \expval{a}{\hat{h}_0}{a} - \expval{i}{\hat{h}_0}{i} + \mathcal{E}_0^{\text{Ref}} \right]
\end{equation*}

For the two-body part, we have that the possible contributing terms must then have an equal number of annihilation and creation operators in order to not vanish, reducing the candidates to
\begin{multicols}{2}{}
    \begin{enumerate}
        \item  $\displaystyle\frac{1}{4} \sum_{abcd} \expval{ab}{V}{cd} b_a^\dagger b_b^\dagger b_d b_c$
        \item  $\displaystyle\frac{1}{2} \sum_{abij} \expval{ai}{V}{bj} b_a^\dagger b_j^\dagger b_b b_i$
        \item  $\displaystyle\frac{1}{2} \sum_{abi} \expval{ai}{V}{bi} b_a^\dagger b_b$
        \item  $\displaystyle\frac{1}{4} \sum_{ijkl} \expval{kl}{V}{ij} b_i^\dagger b_j^\dagger b_l b_k$
        \item  $\displaystyle\frac{1}{2} \sum_{ijk} \expval{ij}{V}{kj} b_k^\dagger b_i$
        \item  $\displaystyle\mathcal{E}_I^\text{Ref}$.
    \end{enumerate}
\end{multicols}

For the terms here, recall that $\wick{\langle c \vert \c1 b_\alpha^\dagger \c1 b_\beta \vert c \rangle} = \delta_{\alpha \beta}$, while $\wick{\langle c \vert \c1 b_\alpha \c1 b_\beta^\dagger \vert c \rangle} = 0$.
The first term vanishes as $b_d b_c\ket{\Phi_j^{b}} =  \underbrace{b_d b_c b_b^\dagger b_j^\dagger}_{0}\ket{c}= 0$, due to over-annihilation of particles.
For the second term, relabeling $(a, b, i, j) \mapsto (c, d, k, l)$ we get
\begin{equation*}
    \wick{
        \langle
        c \vert
        \c2 b_i
        \c1 b_a
        \c1 b_c^\dagger
        \c2 b_l^\dagger
        \c1 b_d
        \c2 b_k
        \c1 b_b^\dagger
        \c2 b_j^\dagger
        \vert c
        \rangle
    } = -\delta_{il} \delta_{ac} \delta_{db} \delta_{kj} \implies
    -\frac{1}{2} \expval{a j}{V}{b i}.
\end{equation*}
For the third term, relabeling $(a, b, i) \mapsto (c, d, k)$, we have
\begin{equation*}
    \wick{
        \langle
        c \vert
        \c2 b_i
        \c1 b_a
        \c1 b_c^\dagger
        \c1 b_d
        \c1 b_b^\dagger
        \c2 b_j^\dagger
        \vert c
        \rangle
    } = \delta_{ij} \delta_{ab} \delta_{cd} \implies
    \frac{1}{2} \delta_{ij} \sum_{k} \expval{a k}{V}{b k}.
\end{equation*}
The fourth term vanishes as the number of hole annihilation operator to the left of the creation operators does not match.
The fifth term, relabeling $(i, j, k) \mapsto (k, l, m)$, gives
\begin{equation*}
    \wick{
        \langle
        c \vert
        \c1 b_i
        \c2 b_a
        \c1 b_m^\dagger
        \c1 b_k
        \c2 b_b^\dagger
        \c1 b_j^\dagger
        \vert c
        \rangle
    } = \delta_{im} \delta_{ab} \delta_{kj} \implies
    \frac{1}{2} \delta_{ab} \sum_{l} \expval{j l}{V}{i l}.
\end{equation*}
The last term simply gives us $\delta_{ab} \delta_{ij} \mathcal{E}_I^{\text{Ref}}$.

In the case of the helium atom, the fifth term vanishes, as there are no remaining holes in the vaccuum states. % TODO: Usikker på om dette er riktig. Er summene a ≠ b typ?
The final expression for $\expval{\Phi_i^a}{\hat{H}}{\Phi_j^b}$ is then
\begin{equation}\label{eq:1p1h-1p1h}
    \begin{split}
        \expval{\Phi_i^a}{\hat{H}}{\Phi_j^b}
        =& \ \delta_{ij} \delta_{ab} \left[
            \expval{a}{\hat{h}_0}{a}
            - \expval{i}{\hat{h}_0}{i}
            + \mathcal{E}_0^{\text{Ref}}
            + \mathcal{E}_I^{\text{Ref}}
        \right] \\
        &- \frac{1}{2} \expval{a j}{V}{b i}
        + \frac{\delta_{ij}}{2} \sum_{k} \expval{a k}{V}{b k}.
    \end{split}
\end{equation}

We split the expression into the different cases.
Firstly, when $i \neq j$ and $a \neq b$, we have
\begin{equation*}
    \expval{\Phi_i^a}{\hat{H}}{\Phi_j^b} = -\frac{1}{2} \expval{aj}{V}{bi}_{AS} = \frac{1}{2} \left[ \expval{aj}{V}{ib} - \expval{aj}{V}{bi} \right].
\end{equation*}
Considering just the spin part, and placing disregarding the energy levels for now, we have
\begin{equation*}
    \frac{1}{2} \left[ \expval{\sigma_1 \sigma_2}{V}{\sigma_1 \sigma_2} - \expval{\sigma_1 \sigma_2}{V}{\sigma_2 \sigma_1} \right],
\end{equation*}
where we can now see that the second state vanishes due to the orthogonality of the spin. % TODO: Dobbeltsjekk at dette stemmer
This leaves us with
\begin{equation*}
    \expval{\Phi_i^a}{\hat{H}}{\Phi_j^b} = \frac{1}{2} \expval{aj}{V}{ib}.
\end{equation*}

Next, we consider the case when $i = j$ and $a \neq b$.
The expression then becomes
\begin{equation*}
    \expval{\Phi_i^a}{\hat{H}}{\Phi_i^b} = \frac{1}{2} \expval{aj}{V}{bj}_{AS} = \frac{1}{2} \left[ \expval{aj}{V}{bj} - \expval{aj}{V}{jb} \right].
\end{equation*}
Again, considering the spin part, we have
\begin{equation*}
    \frac{1}{2} \left[ \expval{\sigma_1 \sigma_2}{V}{\sigma_1 \sigma_2} - \expval{\sigma_1 \sigma_2}{V}{\sigma_2 \sigma_1} \right],
\end{equation*}
where again the right term vanishes, leaving us with
\begin{equation*}
    \expval{\Phi_i^a}{\hat{H}}{\Phi_i^b} = \frac{1}{2} \expval{aj}{V}{bj}.
\end{equation*}
In the case where $i \neq j$ and $a = b$, the entire expression vanishes.

Finally, the expectation value of the Hamiltonian with the one-particle-one-hole excitation is
\begin{equation}\label{eq:1p1h-1p1h-final}
    \expval{\Phi_i^a}{\hat{H}}{\Phi_i^a} = \expval{a}{\hat{h}_0}{a} - \expval{i}{\hat{h}_0}{i} + \mathcal{E}_0^{\text{Ref}} + \mathcal{E}_I^{\text{Ref}}.
\end{equation}

Now, getting the explicit values for the matrix elements, so that we can set up the Hamiltonian matrix and diagonalize it using Python.
