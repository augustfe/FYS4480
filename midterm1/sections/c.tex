Hereafter we will limit ourselves to a system which now contains only one-particle-one-hole excitations beyond the chosen state $\ket{c}$.
Using the possible Slater determinants from exercise a) for the helium atom, find the expressions (without inserting the explicit values for the matrix elements first) for % chktex 10 % tex-fmt: skip
\begin{equation*}
    \expval{c}{\hat{H}}{\Phi_i^a},
\end{equation*}
and
\begin{equation*}
    \expval{\Phi_i^a}{\hat{H}}{\Phi_j^b}.
\end{equation*}
Represent these expressions in a diagrammatic form, both for the onebody part and the two-body part of the Hamiltonian.

Insert then the explicit values for the various matrix elements and set up the final Hamiltonian matrix and diagonalize it using for example Python as programming language.
Compare your results from those of exercise b) and comment your results. % chktex 10 % tex-fmt: skip

The exact energy with our Hamiltonian is $-2.9037$ atomic units for helium.
This value is also close to the experimental energy.

\subsection{}
We start by finding the value of the expression $\expval{c}{\hat{H}}{\Phi_i^a}$.
Writing out the terms, we have
\begin{align*}
    \expval{c}{\hat{H}}{\Phi_i^a} &= \expval{c}{\hat{H}_0 + \hat{H}_I}{\Phi_i^a} \\
    &= \expval{c}{\hat{H}_0}{\Phi_i^a} + \expval{c}{\hat{H}_I}{\Phi_i^a}.
\end{align*}
We can now read from Eq.~\eqref{eq:H0_second_quant} that the one-body part of the expression vanishes, either through an unmatched $b_\alpha^\dagger$ or simply $\braket{c}{\Phi_i^a} = 0$.
This gives us
\begin{equation}\label{eq:1p1h-ground-onebody}
    \expval{c}{\hat{H}_0}{\Phi_i^a} = 0.
\end{equation}

\begin{comment}
    Considering the one-body part, we have
    \begin{align*}
        \expval{c}{\hat{H}_0}{\Phi_i^a} &= \sum_{\alpha\beta} \expval{\alpha}{\hat{h}_0}{\beta} \expval{c}{a_\alpha^\dagger a_\beta}{\Phi_i^a} \\
        &= \sum_{\alpha\beta} \expval{\alpha}{\hat{h}_0}{\beta} \expval{c}{a_\alpha^\dagger a_\beta a_a^\dagger a_i}{c}. \\
    \end{align*}
    The only possible contraction is
    \begin{equation*}
        \langle
        \wick{
            c
            \vert
            \c2 a_\alpha^\dagger \c1 a_\beta \c1 a_a^\dagger \c2 a_i
            \vert
            c
        }
        \rangle,
    \end{equation*}
    such that $\alpha = i$ and $\beta = a$.
    The expectation value of the one-body part is then
    \begin{equation*}
        \expval{c}{\hat{H}_0}{\Phi_i^a} = \sum_{\alpha \beta} \expval{\alpha}{\hat{h}_0}{\beta} \delta_{\alpha i} \delta_{\beta a} = \expval{i}{\hat{h}_0}{a} = 0,
    \end{equation*}
    which vanishes due to the $\delta_{ij}$ term of Eq.~\eqref{eq:onebody}.
\end{comment}

In order to make sure the number of annihilation and creation terms are correct, the contributing terms from $H_I$ are those with two more annihilation operators than creation operators, while also matching the number of holes and particles created.
We can then quickly reduce the possible contributing candidates to
\begin{equation*}
    \frac{1}{4} \sum_{aijk} \expval{ji}{V}{ak} b_k^\dagger b_j b_i b_a, \qquad
    \frac{1}{2} \sum_{aij} \expval{ji}{V}{ai} b_j b_a, \qquad
    \text{and} \qquad \mathcal{E}_I^\text{Ref}.
\end{equation*}
On closer inspection, we see that the first term vanishes, as $b_k^\dagger$ is unmatched, while the last term vanished due to $\braket{c}{\Phi_i^a} = 0$.

We then just need to evaluate the contractions of the second term, changing the labels of the sum $(a, i, j) \mapsto (b, j, k)$ to avoid confusion with $\ket{\Phi_i^a}$, which simply is
\begin{equation*}
    \langle c \vert
    \wick{
        \c2 b_k \c1 b_b \c1 b_a^\dagger \c2 b_i^\dagger
    }
    \vert c \rangle
    = \delta_{ik} \delta_{ab}.
\end{equation*}
We then get
\begin{equation}\label{eq:1p1h-ground-twobody}
    \expval*{c}{\hat{H}_I}{\Phi_i^a} = \frac{1}{2} \sum_{j} \expval{i j}{V}{a j}_{AS},
\end{equation}
taking into account that the matrix element in the sum is antisymmetrized.
Combining Eqs.~\eqref{eq:1p1h-ground-onebody}~and~\eqref{eq:1p1h-ground-twobody}, we get the final expression is
\begin{equation}\label{eq:1p1h-ground}
    \expval{c}{\hat{H}}{\Phi_i^a} = \frac{1}{2} \sum_{j \neq i} \expval{i j}{V}{a j} - \expval{i j}{V}{j a}.
\end{equation}

Finally, taking into account spin, the values we get for the one-particle-one-hole excitations are listed in Eq.~\eqref{eq:1p1h-ground-list}, with the only possible value for $j$ in each case of Eq.~\eqref{eq:1p1h-ground} listed to the far left.
For brevity, we write $j\sigma_{\pm}$ as just $j_{\pm}$ here.
\begin{equation}\label{eq:1p1h-ground-list}
    \arraycolsep=1.4pt
    \begin{array}{crcl}
        j = 1_{-}: \quad
        & \expval*{c}{\hat{H}}{\Phi_{1_{+}}^{2_{+}}}\hspace{0.27pt}
        & = &
        \expval{1_{-} 1_{+}}{V}{1_{-} 2_{+}} - \expval{1_{-} 1_{+}}{V}{2_{+} 1_{-}} \\

        j = 1_{-}: \quad
        & \expval*{c}{\hat{H}}{\Phi_{1_{+}}^{2_{+}}}\hspace{0.27pt}
        & = &
        \expval{1_{-} 1_{+}}{V}{1_{-} 3_{+}} - \expval{1_{-} 1_{+}}{V}{3_{+} 1_{-}} \\

        j = 1_{+}: \quad
        & \expval*{c}{\hat{H}}{\Phi_{1_{-}}^{2_{-}}}
        & = &
        \expval{1_{+} 1_{-}}{V}{1_{+} 2_{-}} - \expval{1_{+} 1_{-}}{V}{2_{-} 1_{+}} \\

        j = 1_{+}: \quad
        & \expval*{c}{\hat{H}}{\Phi_{1_{-}}^{3_{-}}}
        & = &
        \expval{1_{+} 1_{-}}{V}{1_{+} 3_{-}} - \expval{1_{+} 1_{-}}{V}{3_{-} 1_{+}}
    \end{array}
\end{equation}

\begin{comment}
    \newpage

    For the two-body part, writing $V$ for the two-particle operator, we have
    \begin{align*}
        \expval{c}{\hat{H}_I}{\Phi_i^a} &= \frac{1}{2} \sum_{\alpha\beta\gamma\delta} \expval{\alpha\beta}{V}{\gamma\delta} \expval{c}{a_\alpha^\dagger a_\beta^\dagger a_\delta a_\gamma}{\Phi_i^a} \\
        &= \frac{1}{2} \sum_{\alpha\beta\gamma\delta} \expval{\alpha\beta}{V}{\gamma\delta} \expval{c}{a_\alpha^\dagger a_\beta^\dagger a_\delta a_\gamma a_a^\dagger a_i}{c}.
    \end{align*}
    The possible contractions are then
    \begin{align*}
        \expval{\alpha\beta}{V}{\gamma\delta} \langle
        \wick{
            c
            \vert
            \c3 a_\alpha^\dagger \c2 a_\beta^\dagger \c1 a_\delta \c3 a_\gamma \c1 a_a^\dagger \c2 a_i
            \vert
            c
        }
        \rangle
        &= \delta_{\alpha \gamma} \delta_{\beta i} \delta_{\delta a} \expval{\alpha\beta}{V}{\gamma\delta}
        = \expval{\alpha i}{V}{\alpha a}, \\
        \expval{\alpha\beta}{V}{\gamma\delta} \langle
        \wick{
            c
            \vert
            \c3 a_\alpha^\dagger \c2 a_\beta^\dagger \c3 a_\delta \c1 a_\gamma \c1 a_a^\dagger \c2 a_i
            \vert
            c
        }
        \rangle
        &= -\delta_{\alpha \delta} \delta_{\beta i} \delta_{\gamma a} \expval{\alpha\beta}{V}{\gamma\delta} = -\expval{\alpha i}{V}{a \alpha}, \\
        \expval{\alpha\beta}{V}{\gamma\delta} \langle
        \wick{
            c
            \vert
            \c3 a_\alpha^\dagger \c2 a_\beta^\dagger \c1 a_\delta \c2 a_\gamma \c1 a_a^\dagger \c3 a_i
            \vert
            c
        }
        \rangle
        &= -\delta_{\alpha i} \delta_{\beta \gamma} \delta_{a \delta} \expval{\alpha\beta}{V}{\gamma\delta} = -\expval{i \beta}{V}{\beta a}, \\
        \expval{\alpha\beta}{V}{\gamma\delta} \langle
        \wick{
            c
            \vert
            \c2 a_\alpha^\dagger \c1 a_\beta^\dagger \c1 a_\delta \c1 a_\gamma \c1 a_a^\dagger \c2 a_i
            \vert
            c
        }
        \rangle
        &= \delta_{\alpha i} \delta_{\beta \delta} \delta_{\gamma a} \expval{\alpha\beta}{V}{\gamma\delta} = \expval{i \beta}{V}{a \beta}.
    \end{align*}
    Using the general fact that $\expval{\alpha \beta}{V}{\gamma \delta} = \expval{\beta \alpha}{V}{\delta \gamma}$, we can gather these terms into a single term
    % TODO: FIX OSKAR, gjør greia di
    % TODO: Commutation relations
    % TODO: Diagrams
    \begin{equation*}
        \expval{c}{\hat{H}_I}{\Phi_i^a} = \sum_{\alpha} \expval{\alpha i}{V}{\alpha a} - \expval{\alpha i}{V}{a \alpha}.
    \end{equation*}

    The final expression for the expectation value of the Hamiltonian between the ground state and a one-particle-one-hole excitation is then
    \begin{equation}\label{eq:1p1h-exp}
        \expval{c}{\hat{H}}{\Phi_i^a} = \expval{c}{\hat{H}_0}{\Phi_i^a} + \expval{c}{\hat{H}_I}{\Phi_i^a} = 0 + \sum_{\alpha} \expval{\alpha i}{V}{\alpha a} - \expval{\alpha i}{V}{a \alpha}.
    \end{equation}

    Finally, the values we get for the one-particle-one-hole excitations are listed in Eq.~\eqref{eq:1p1h-exp-list}, with the only possible value for $\alpha$ in each case of Eq.~\eqref{eq:1p1h-exp} listed to the far left.
    For brevity, we write $\beta\sigma_{\pm}$ as just $\beta_{\pm}$ here.
    \begin{equation}\label{eq:1p1h-exp-list}
        \arraycolsep=1.4pt % TODO: Fiks off-by-one pixel Oskar
        \begin{array}{crcl}
            \alpha = 1_{-}: \quad
            & \expval*{c}{\hat{H}}{\Phi_{1_{+}}^{2_{+}}}
            & = &
            \expval{1_{-} 1_{+}}{V}{1_{-} 2_{+}} - \expval{1_{-} 1_{+}}{V}{2_{+} 1_{-}} \\

            \alpha = 1_{-}: \quad
            & \expval*{c}{\hat{H}}{\Phi_{1_{+}}^{2_{+}}}
            & = &
            \expval{1_{-} 1_{+}}{V}{1_{-} 3_{+}} - \expval{1_{-} 1_{+}}{V}{3_{+} 1_{-}} \\

            \alpha = 1_{+}: \quad
            & \expval*{c}{\hat{H}}{\Phi_{1_{-}}^{2_{-}}}
            & = &
            \expval{1_{+} 1_{-}}{V}{1_{+} 2_{-}} - \expval{1_{+} 1_{-}}{V}{2_{-} 1_{+}} \\

            \alpha = 1_{+}: \quad
            & \expval*{c}{\hat{H}}{\Phi_{1_{-}}^{3_{-}}}
            & = &
            \expval{1_{+} 1_{-}}{V}{1_{+} 3_{-}} - \expval{1_{+} 1_{-}}{V}{3_{-} 1_{+}}
        \end{array}
    \end{equation}
\end{comment}

Next, we find a simplified expression for $\expval{\Phi_i^a}{\hat{H}}{\Phi_j^b}$,
noting that
\begin{equation*}
    \expval{\Phi_i^a}{\hat{H}}{\Phi_j^b} = \expval{c}{b_i b_a \hat{H} b_b^\dagger b_j^\dagger}{c}.
\end{equation*}
The possible contributing terms must then have an equal number of annihilation and creation operators in order to not vanish, reducing the candidates to
% \begin{enumerate}
%     \item $\displaystyle\frac{1}{4} \sum_{abcd} \expval{ab}{V}{cd} b_a^\dagger b_b^\dagger b_d b_c$
%     \item $\displaystyle\frac{1}{2} \sum_{abij} \expval{ai}{V}{bj} b_a^\dagger b_j^\dagger b_b b_i$
%     \item $\displaystyle\frac{1}{2} \sum_{abi} \expval{ai}{V}{bi} b_a^\dagger b_b$
%     \item $\displaystyle\frac{1}{4} \sum_{ijkl} \expval{kl}{V}{ij} b_i^\dagger b_j^\dagger b_l b_k$
%     \item $\displaystyle\frac{1}{2} \sum_{ijkl} \expval{ij}{V}{kj} b_k^\dagger b_i$
%     \item $\displaystyle\mathcal{E}_I^\text{Ref}$
% \end{enumerate}
\begin{equation*} % TODO: Fix formattering Oskar
    \begin{array}{cccccc}
        \displaystyle\frac{1}{4} \sum_{abcd} \expval{ab}{V}{cd} b_a^\dagger b_b^\dagger b_d b_c &, &
        \displaystyle\frac{1}{2} \sum_{abij} \expval{ai}{V}{bj} b_a^\dagger b_j^\dagger b_b b_i &, &
        \displaystyle\frac{1}{2} \sum_{abi} \expval{ai}{V}{bi} b_a^\dagger b_b, \\
        \displaystyle\frac{1}{4} \sum_{ijkl} \expval{kl}{V}{ij} b_i^\dagger b_j^\dagger b_l b_k &, &
        \displaystyle\frac{1}{2} \sum_{ijk} \expval{ij}{V}{kj} b_k^\dagger b_i &, &
        \displaystyle\mathcal{E}_I^\text{Ref}.
    \end{array}
\end{equation*}

% TODO: Bra nok formattering?
\begin{multicols}{2}{}
    \begin{enumerate}
        \item  $\displaystyle\frac{1}{4} \sum_{abcd} \expval{ab}{V}{cd} b_a^\dagger b_b^\dagger b_d b_c$
        \item  $\displaystyle\frac{1}{2} \sum_{abij} \expval{ai}{V}{bj} b_a^\dagger b_j^\dagger b_b b_i$
        \item  $\displaystyle\frac{1}{2} \sum_{abi} \expval{ai}{V}{bi} b_a^\dagger b_b$
        \item  $\displaystyle\frac{1}{4} \sum_{ijkl} \expval{kl}{V}{ij} b_i^\dagger b_j^\dagger b_l b_k$
        \item  $\displaystyle\frac{1}{2} \sum_{ijk} \expval{ij}{V}{kj} b_k^\dagger b_i$
        \item  $\displaystyle\mathcal{E}_I^\text{Ref}.$
    \end{enumerate}
\end{multicols}

For the terms here, recall that $\wick{\langle c \vert \c1 b_\alpha^\dagger \c1 b_\beta \vert c \rangle} = \delta_{\alpha \beta}$, while $\wick{\langle c \vert \c1 b_\alpha \c1 b_\beta^\dagger \vert c \rangle} = 0$.
The first term vanishes as $b_d b_c\ket{\Phi_j^{b}} =  \underbrace{b_d b_c b_b^\dagger b_j^†}_{0}\ket{c}= 0$, due to over-annihilation of particles.
For the second term, relabeling $(a, b, i, j) \mapsto (c, d, k, l)$ we get
\begin{equation*}
    \wick{
        \langle
        c \vert
        \c2 b_i
        \c1 b_a
        \c1 b_c^\dagger
        \c2 b_l^\dagger
        \c1 b_d
        \c2 b_k
        \c1 b_b^\dagger
        \c2 b_j^\dagger
        \vert c
        \rangle
    } = -\delta_{il} \delta_{ac} \delta_{db} \delta_{kj} \implies \frac{1}{2} \sum
\end{equation*}
\begin{comment}
    \begin{align*}
        \wick{
            \langle
            c \vert
            \c2 b_i
            \c1 b_a
            \underbrace{
                b_c^\dagger
                b_l^\dagger
                b_d
                b_k
            }_{\text{vanishes}}
            \c1 b_b^\dagger
            \c2 b_j^\dagger
            \vert c
            \rangle
        } &= 0, &
        \wick{
            \langle
            c \vert
            \c2 b_i
            \c1 b_a
            \c1 b_c^\dagger
            \c2 b_l^\dagger
            \c1 b_d
            \c2 b_k
            \c1 b_b^\dagger
            \c2 b_j^\dagger
            \vert c
            \rangle
        } &= -\delta_{il} \delta_{ac} \delta_{db} \delta_{kj}.
    \end{align*}
\end{comment}
For the third term, relabeling $(a, b, i) \mapsto (c, d, k)$, we have
\begin{equation*}
    \wick{
        \langle
        c \vert
        \c2 b_i
        \c1 b_a
        \c1 b_c^\dagger
        \c1 b_d
        \c1 b_b^\dagger
        \c2 b_j^\dagger
        \vert c
        \rangle
    } = \delta_{ij} \delta_{ab} \delta_{cd}.
\end{equation*}
The fourth term vanishes as the number of hole annihilation operator to the left of the creation operators does not match.
The fifth term, relabeling $(i, j, k) \mapsto (k, l, m)$, gives
\begin{equation*}
    \wick{
        \langle
        c \vert
        \c1 b_i
        \c2 b_a
        \c1 b_m^\dagger
        \c1 b_k
        \c2 b_b^\dagger
        \c1 b_j^\dagger
        \vert c
        \rangle
    } = \delta_{im} \delta_{ab} \delta_{kj}.
\end{equation*}
The last term simply gives us $\delta_{ab} \delta_{ij}$.

\newpage

Next, we find the expectation value of the Hamiltonian between two one-particle-one-hole excitations.
Considering the one-body part, we have
\begin{align*}
    \expval{\Phi_i^a}{\hat{H}_0}{\Phi_j^b} &= \sum_{\alpha\beta} \expval{\alpha}{\hat{h}_0}{\beta} \expval{\Phi_i^a}{a_\alpha^\dagger a_\beta}{\Phi_j^b} \\
    &= \sum_{\alpha\beta} \expval{\alpha}{\hat{h}_0}{\beta} \expval{c}{a_i^\dagger a_a  a_\alpha^\dagger a_\beta a_b^\dagger a_j}{c}.
\end{align*}
Note that if $\alpha > F$ or $\beta > F$, the expectation value vanishes.

The contractions are then
\begin{equation*}
    \arraycolsep=1.4pt
    \begin{array}{rlrcr}
        \expval{\alpha}{\hat{h}_0}{\beta}
        \langle
        \wick{
            c
            \vert
            \c2 a_i^\dagger \c1 a_a \c1 a_\alpha^\dagger \c1 a_\beta \c1 a_b^\dagger \c2 a_j
            \vert
            c
        }
        \rangle
        &=& \delta_{ij} \delta_{a \alpha} \delta_{b \beta} \expval{\alpha}{\hat{h}_0}{\beta}
        &=& \delta_{ij} \expval{a}{\hat{h}_0}{b} \,
        \\
        \expval{\alpha}{\hat{h}_0}{\beta}
        \langle
        \wick{
            c
            \vert
            \c1 a_i^\dagger \c2 a_a \c3 a_\alpha^\dagger \c1 a_\beta \c2 a_b^\dagger \c3 a_j
            \vert
            c
        }
        \rangle
        &= &- \delta_{i \beta} \delta_{a b} \delta_{\alpha j} \expval{\alpha}{\hat{h}_0}{\beta}
        &=& -\delta_{ab} \expval{j}{\hat{h}_0}{i} \;  \\
        \expval{\alpha}{\hat{h}_0}{\beta}
        \langle
        \wick{
            c
            \vert
            \c3 a_i^\dagger \c2 a_a \c1 a_\alpha^\dagger \c1 a_\beta \c2 a_b^\dagger \c3 a_j
            \vert
            c
        }
        \rangle
        &=& \delta_{i j} \delta_{a b} \delta_{\alpha \beta} \expval{\alpha}{\hat{h}_0}{\beta}
        &=& \delta_{i j} \delta_{a b} \expval{\alpha}{\hat{h}_0}{\alpha} \\
    \end{array}
\end{equation*}
meaning that the expectation value of the one-body part is
\begin{equation*}
    \expval{\Phi_i^a}{\hat{H}_0}{\Phi_j^b} = \delta_{ij} \expval{a}{\hat{h}_0}{b} - \delta_{ab} \expval{j}{\hat{h}_0}{i} + \delta_{ij} \delta_{ab} \sum_{\alpha \leq F} \expval{\alpha}{\hat{h}_0}{\alpha},
\end{equation*}
where we recognize the right-most term as the expectation value of the one-body operator in the ground state, known as the reference energy $\mathcal{E}_0^{\text{Ref}}$.
The term then vanishes whenever $i \neq j$ or $a \neq b$.

For the two-body part, we have
\begin{align*}
    \expval{\Phi_i^a}{\hat{H}_I}{\Phi_j^b} &= \frac{1}{2} \sum_{\alpha\beta\gamma\delta} \expval{\alpha\beta}{V}{\gamma\delta} \expval{\Phi_i^a}{a_\alpha^\dagger a_\beta^\dagger a_\delta a_\gamma}{\Phi_j^b} \\
    &= \frac{1}{2} \sum_{\alpha\beta\gamma\delta} \expval{\alpha\beta}{V}{\gamma\delta} \expval{c}{a_i^\dagger a_a a_\alpha^\dagger a_\beta^\dagger a_\delta a_\gamma a_b^\dagger a_j}{c}.  \\
\end{align*}
The possible contractions are then

